\documentclass[a4paper, 12pt]{article}

%Paragraph jumps and indentation
\setlength{\parskip}{1.6em}
\setlength{\parindent}{1cm}

%Border
\usepackage[left=1in, right=1in, top=1in, bottom=1in]{geometry}

%Double spacing
\usepackage{setspace}
\doublespacing
%Packages
\usepackage{amsmath}
\usepackage[dvipsnames]{xcolor}
\usepackage{mathtools}
\usepackage{amsfonts}
\usepackage{titlesec}

%Images
\usepackage{graphicx}
\graphicspath{ {./images/} }
\usepackage{wrapfig}
\usepackage{float}

%Tables
\usepackage{multirow}
\usepackage{array}
\usepackage{tabu}
\titleformat{\section}
{\normalfont\large\bfseries}{\thesection}{1em}{}
\titleformat{\subsection}
{\normalfont\large\bfseries}{\thesubsection}{1em}{}

%Equation numbering
\counterwithin{equation}{section}

%links
\usepackage{hyperref}
\hypersetup{
    colorlinks=true,
    linkcolor=blue,
    filecolor=magenta,      
    urlcolor=cyan,
    pdftitle={Overleaf Example},
    pdfpagemode=FullScreen,
    }
\urlstyle{same}

%Diagrams
\usepackage{pgfplots}
\pgfplotsset{compat=newest}
\usetikzlibrary{positioning, arrows.meta}
\usepgfplotslibrary{fillbetween}

%Theorems
\newtheorem{theorem}{Theorem}[section]
\newtheorem{definition}{Definition}[section]
\newtheorem{lemma}{Lemma}[section]
\newtheorem{corollary}{Corollary}[section]

%Bibliography
\usepackage[backend=biber]{biblatex}
\addbibresource{refs.bib}

\begin{document}

\begin{titlepage}
  \begin{center}
    Tok Essay\\
    \vspace*{5cm}    
    {\large \textbf{Prompt 3.}\\
    Is the power of knowledge determined by the way in which the knowledge is conveyed?
    Discuss with reference to mathematics \textbf{and one other} area of knowledge.}
    \vfill
    Word count: 0000
  \end{center}
\end{titlepage}
\section*{Plan.}
My other area of knowledge is human sciences. (?)

Define 'power of knowledge': The influence that knowledge has on the world.
Ex: Historical knowledge may be more powerful, i.e. recounted more, if it was written in a more aesthetic way.

Reading / material for real life examples:
\begin{itemize}
  \item A Mathematician's Apology
  \item Thinking, Fast and Slow (Nudge)
  \item Philosophy of Mathematics and Natural Sciences
  \item Mathematics, Queen and Servant of Science
\end{itemize}
Hopefully this has enough material.

Then my other area of knowledge, Human Sciences, would be economics or psychology...

I'll begin work on the essay when I've read the works

\section{Introduction}
hallo

\newpage
\printbibliography

\end{document}