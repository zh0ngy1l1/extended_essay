\documentclass[a4paper, 12pt]{article}

%Paragraph jumps and indentation
\setlength{\parskip}{1.6em}
\setlength{\parindent}{1cm}

%Border
\usepackage[left=1in, right=1in, top=1in, bottom=1in]{geometry}

%Double spacing
\usepackage{setspace}
\doublespacing
%Packages
\usepackage{amsmath}
\usepackage[dvipsnames]{xcolor}
\usepackage{mathtools}
\usepackage{amsfonts}
\usepackage{titlesec}

%Images
\usepackage{graphicx}
\graphicspath{ {./images/} }
\usepackage{wrapfig}
\usepackage{float}

%Tables
\usepackage{multirow}
\usepackage{array}
\usepackage{tabu}
\titleformat{\section}
{\normalfont\large\bfseries}{\thesection}{1em}{}
\titleformat{\subsection}
{\normalfont\large\bfseries}{\thesubsection}{1em}{}

%Equation numbering
\counterwithin{equation}{section}

%links
\usepackage{hyperref}
\hypersetup{
    colorlinks=true,
    linkcolor=blue,
    filecolor=magenta,      
    urlcolor=cyan,
    pdftitle={Overleaf Example},
    pdfpagemode=FullScreen,
    }
\urlstyle{same}

%Diagrams
\usepackage{pgfplots}
\pgfplotsset{compat=newest}
\usetikzlibrary{positioning, arrows.meta}
\usepgfplotslibrary{fillbetween}

%Theorems
\newtheorem{theorem}{Theorem}[section]
\newtheorem{definition}{Definition}[section]
\newtheorem{lemma}{Lemma}[section]
\newtheorem{corollary}{Corollary}[section]

%Bibliography
\usepackage[backend=biber]{biblatex}
\addbibresource{refs.bib}

\begin{document}

\begin{titlepage}
  \begin{center}
    Tok Essay\\
    \vspace*{5cm}    
    {\large \textbf{Prompt 3.}\\
    Is the power of knowledge determined by the way in which the knowledge is conveyed?
    Discuss with reference to mathematics \textbf{and one other} area of knowledge.}
    \vfill
    Word count: 0000
  \end{center}
\end{titlepage}
\section*{Plan.}
My other area of knowledge is human sciences. (?)

Define 'power of knowledge': The influence that knowledge has on the world.
Ex: Historical knowledge may be more powerful, i.e. recounted more, if it was written in a more aesthetic way.

Reading / material for real life examples:
\begin{itemize}
  \item A Mathematician's Apology
  \item Thinking, Fast and Slow (maybe Nudge as well?)
  \item Philosophy of Mathematics and Natural Sciences
  \item Mathematics, Queen and Servant of Science
\end{itemize}
Hopefully this has enough material.

Then my other area of knowledge, Human Sciences, would relate to economics or psychology...

Conjectures: 

\textbf{The way in which knowledge is conveyed:} From a figure of authority.
Authority is useless (or has very little use) in the power of mathematical knowledge.
Fermat, the greatest mathematician at his time, claims he has proven his last theorem $a^n + b^n = c^n$.
Yet, before Wiles's proof and verification, the theorem is not treated as much more plausible than, say, nonexistence of odd perfect numbers.

\textbf{The way in which knowledge is conveyed:} motivation, salience... 

A \textit{model} is a set of rules, abstractions, simplifications, and theory, through which knowledge may be interpreted.

Descartes' invention of algebraic geometry:
\begin{quote}
  
\end{quote}


mathematical proof vs empirical evidence

\section{Introduction}
hallo
 
\newpage
\printbibliography

\end{document}

%The way in which knowledge is conveyed; Not just the people, but also other means. Story, equation, written word/spoken word
%Counterexample for maths

%ex, ctex for the other direction too.

% How history is used by/not used by policymakers.
% to historians and scholars, decisions may seem ignorant, but there really is a gap between scholars and Leaders, likely because how knowledge is conveyed.
% It's not enough that knowledge exists somehere; conveying it is equally critical for the power of this knowledge to manifest. 
% Maybe Gapminder: Psychology?
% Shows how important the means of conveying knowledge is. 
% since these statistics of *stuff* always exists somewhere on the internet and it just has to be shown to people.

% Assuming I take this, I just need to figure out an argument against. 
% So why it MAY NOT BE that important.
% " Personal things? "

% CHATGPT:
% Example (Psychology): Classical Conditioning Works Regardless of Medium
% Alternative Psychology Example: Placebo Effect Does Not Depend on Medium
% Another Option: Memory Reconsolidation