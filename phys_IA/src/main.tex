\documentclass[a4paper, 12pt]{article}

%Paragraph jumps and indentation
\setlength{\parskip}{1.6em}
\setlength{\parindent}{1cm}

%Border
\usepackage[left=1in, right=1in, top=1in, bottom=1in]{geometry}

%Double spacing
\usepackage{setspace}
\doublespacing
%Packages
\usepackage{amsmath}
\usepackage[dvipsnames]{xcolor}
\usepackage{mathtools}
\usepackage{amsfonts}
\usepackage{titlesec}

%Images
\usepackage{graphicx}
\graphicspath{ {./images/} }
\usepackage{wrapfig}
\usepackage{float}

%Tables
\usepackage{multirow}
\usepackage{array}
\usepackage{tabu}
\titleformat{\section}
{\normalfont\large\bfseries}{\thesection}{1em}{}
\titleformat{\subsection}
{\normalfont\large\bfseries}{\thesubsection}{1em}{}

%Equation numbering
\counterwithin{equation}{section}

%links
\usepackage{hyperref}
\hypersetup{
    colorlinks=true,
    linkcolor=blue,
    filecolor=magenta,      
    urlcolor=cyan,
    pdftitle={Overleaf Example},
    pdfpagemode=FullScreen,
    }
\urlstyle{same}

%Diagrams
\usepackage{pgfplots}
\pgfplotsset{compat=newest}
\usetikzlibrary{positioning, arrows.meta}
\usepgfplotslibrary{fillbetween}

%Theorems
\newtheorem{theorem}{Theorem}[section]
\newtheorem{definition}{Definition}[section]
\newtheorem{lemma}{Lemma}[section]
\newtheorem{corollary}{Corollary}[section]

%Bibliography
\usepackage[backend=biber]{biblatex}
\addbibresource{refs.bib}

\begin{document}

\begin{titlepage}
  \begin{center}
    Internal Assessment: Physics\\
    \vspace*{4cm}
    {\Large \textbf{Computing the dynamic frictional coefficient $\mu_d$ between wood and steel by measuring the relationship between the force of friction and mass.}}\\
    
    \vspace{1cm}
    \textbf{Research Question:}
    How does changing the mass of a box sliding against a plane affect the force of friction it experiences?
    \vfill
    Word count: 1392
  \end{center}
\end{titlepage}

\section{Introduction}
It is perplexing how the force of friction a sliding object
experiences happens to be proportional to the size of the normal force at
the surface it slides against.
We perhaps take it for granted that so many physical phenomena may be described with
a proportional model, yet this fact strikes the author as rather remarkable.
Despite this, attempting to answer \textit{why} proportionality describes the behaviour
of friction is far beyond the scope of this discussion.
Instead, we shall merely satisfy our curiousity with an example from
which it is easy to observe that the proportional relationship is indeed
present.

\section{Variables}
The independent variable is the mass of the box, which is varied in approximately
$200g$ increments.
The dependent variable is the frictional force that the box experiences
when sliding against the plane.
This is measured using a Vernier dual range force sensor.
The speed at which the box slides is a notable control variable.
To mitigate errors arising from movement at different speeds, the experiment
is performed thrice for each mass, with the median of the results selected
for further analysis. The maximum and minimum over all three runs
are used for error estimation.

\section{Theory}
It is known that force of dynamic friction of a body sliding against a
surface is proportional to the normal force that the surface exerts upon
the body.
Theoretically, this force is not dependent on the area of the contact surface
between the body and the surface, nor the speed of the body.
This implies the existence of a constant, conventionally denoted by $\mu_d$
and called the `dynamic frictional coefficient', which is only dependent
on the materials which are sliding against each other, and reports the
relation between the normal force and the frictional force.
Therefore, in this experiment it suffices to compute the size of the normal
force exerted onto the box given its mass, and compare it against the 
measured average friction.
Since the box and the plane are level, the normal force exerted by the plane
onto the box is equal to the force of gravitational attraction that the box
experiences.
Formally, if $g$ is the gravitational constant ($\approx 9.81$ m s$^{-2}$) and
$m$ is the mass of the box, the normal force $R$ can be calculated as follows:
\begin{equation}
  R=G=m\cdot g
\end{equation}
The coefficient of friction $\mu_d$ is defined as
\begin{equation}
  \mu_d = \frac{F_\mu}{R}
\end{equation}
In terms of data processing, the average frictional force of the sliding
body $F_\mu$ can be measured directly by reading the force sensor.
The normal force $R$ exerted on to the box can be computed given its
mass using (3.1).

\newpage


\section{Existing Results}
\vspace{-2em}\begin{wrapfigure}{l}{0.37\textwidth}
  \begin{center}
    \includegraphics[width=0.36\textwidth]{boxplane.png}
  \end{center}
  \caption{Box and plane}
\end{wrapfigure}
In this experiment, the plane is a plank of presumably fir or pine.
The box is wooden, but it makes contact with the plane with some metal
railings, presumably some iron alloy like stainless steel.
The box, with its metal railings, is visible in \textbf{Figure 1}, with
the wooden plane in the background.

There were no existing results regarding the coefficient of dynamic friction 
with this fir or pine sliding against metal readily available.
However, trials using western hemlock should suffice as a reference.
After all, both woods are from trees belonging to the \textit{Pinaceae}
family and are used in construction.

According to Murase's study\cite{Murase1984}, the dynamic frictional coefficient is approximately $1.2$ for stainless steel and $1.6$ for mild steel against western hemlock wood.
The following differences in the methodology may explain a greater observed frictional coefficient in this investigation compared to Murase's existing study:

\begin{center}
\renewcommand{\arraystretch}{1.5}
\begin{table}[h]
  \begin{tabular}{p{0.45\textwidth} p{0.45\textwidth}}
  \hline
  \textbf{Existing study} & \textbf{This investigation} \\
  \hline
  Uses air-dried wood and shows that increasing moisture content increases friction & Wood moisture content not considered \\
  Uses clean steel with documented surface roughness & Steel railings are slightly rusted (\textbf{Figure 1}) \\
  Uses western hemlock & Uses softer woods such as European fir or pine \\
  \hline
  \end{tabular}
\end{table}
\end{center}

\section{Apparatus}
The apparatus in this investigation is very similar to the one used in
Murase's study.
\begin{quote}
  \hspace{-1em}\vrule width 1pt\hspace{1em}%
  \begin{minipage}[t]{0.95\linewidth}
    \begin{center}
        \includegraphics[width=0.6\textwidth]{murase_apparatus.png}

    \end{center}

    Figure 2. Schematic representation of test apparatus.
    \textcircled{1} Wood specimen,
    \textcircled{2} Counterface material,
    \textcircled{3} Load cell,
    \textcircled{4} Weigh,
    \textcircled{5} Carriage.
    
    \hfill {\color{gray} \footnotesize Copied from \cite{Murase1984}}
  \end{minipage}
\end{quote}
Fewer parts are used in the apparatus of this investigation, as seen in figure \textbf{Figure 3} below.
\textcircled{1} and \textcircled{4} are collectively referred to as the `box', whose mass varies by changing the number of weights set on the sliding block.
The plank that is pushed acts simultaneously as both the \textcircled{2} counterface material and the \textcircled{5} carriage. 
\textcircled{3} is the Vernier force sensor, also called load cell, used for data collection.

\setcounter{figure}{2}
\begin{figure}[h]
  \centering
  \begin{tikzpicture}
    \def\imgwidth{12.96cm}
    \def\imgheight{8.19cm}
    
    \node[anchor=south west,inner sep=0] (img) at (0,0)
      {\includegraphics[width=\imgwidth,height=\imgheight]{trial.png}};
    
    \begin{scope}[x=1.5cm/100, y=1.5cm/100]

      \draw[white,fill] (270,101) circle (15);
      \node at (270,100) {\textcircled{1}};

      \draw[white,fill] (700,481) circle (15);
      \node at (700,480) {\textcircled{3}};

      \draw[white,fill] (200,301) circle (15);
      \node at (200,300) {\textcircled{4}};

      \draw[white,fill,rounded corners=7] 
      (476-15,201-15) rectangle (524+15,201+15);
      \node at (500,200) {\textcircled{2}, \textcircled{5}};
    \end{scope}
  \end{tikzpicture}
  \caption{Photo of apparatus}
\end{figure}
The force sensor is secured to an unmoving platform.
The box and the force sensor are connected with a piece of string, selected to minimize
unwanted unwanted behaviour.
The mass is varied in $200$g increments, by varying the number of weights on the box.

\section{procedure}
\textbf{For each trial:}
Put the desired amount of weight onto the box, and measure with a scale to verify.
Perform at least three runs for each value of mass tested.

\textbf{For each run:}
Have the string slack and running parallel with the grain of the wood and in the effective direction of the force sensor.
Start data collection and attempt to push the plank at a constant speed, directly away from the force sensor.

\section{Data and processing}
This investigation begins at no weights ($m=195$g) and proceeds to increment till 6 weights ($m=1495$g) are on the box.
In each trial, three runs are performed and their average frictions measured between $t=6$s and $t=8$s are recorded.
Of the three averages, the median is selected for further data processing.
The maximum and minimum frictions recorded across these three runs in this time interval are used for error estimation.

\newpage
\begin{figure}[h]
  \centering
  \includegraphics[width=0.8\textwidth]{data_06.png}
  \caption{Measurements for $m=795$g}
\end{figure}
For example, in \textbf{Figure 4}, the median average friction, which is the one selected for further analysis, is $1.54$N. (Labeled `Auto Fit for Run 3')
The maximum friction (in red at $t\approx 7.97$s) is $1.66$N
and the minimum friction (in green at $t \approx 7.64$s) is $1.42$N.
This gives an uncertainty of $\frac{1.66-1.42}{2}=0.12$N for friction in this trial.

The mass of the box in this trial is $995$g$= 0.995$kg, so the normal force is $0.995\cdot9.81\approx9.76$N.
The uncertainty associated with the normal force is a constant $0.01$N, arising from the scale having a precision of $1$g.

In this manner, the average friction and its uncertainty are deduced and computed for each trial.
The normal force and its uncertainty are also computed, giving \textbf{Table 1} below.
\begin{table}[h]
  \centering
  \begin{tabular}{|c|c|} 
   \hline
   $R$ (N)& $F_\mu$ (N)\\ 
   \hline
   $1.91$ & $0.36\pm 0.12$\\
   $3.87$ & $0.74\pm 0.13$\\
   $5.84$ & $1.14\pm 0.12$\\
   $7.80$ & $1.54\pm 0.12$\\
   $9.76$ & $1.79\pm 0.25$\\
   $11.73$& $2.13\pm 0.21$\\
   $13.68$& $2.48\pm 0.17$\\
   \hline
  \end{tabular}
  \caption{Processed data}
\end{table}

\section{Evaluations}
We plot the processed data and associated uncertainties on the $(F_\mu,R)$ plane.
Observation of a proportional relationship would serve as evidence that the force of friction experienced by the box is proportional to the normal force exerted on it by the plane.
In that case we call the proportionality constant the `frictional coefficient' between the materials, wood and steel in this case.
\begin{figure}[h]
  \centering
  \includegraphics[width=0.8\textwidth]{Screenshot 2025-10-03 155223.png}
  \caption{Fitted curves}
\end{figure}
Note that here the horizontal error bars are (by mistake) set to 0.1N instead of 0.01N,
exaggerating the uncertainty of mass, and subsequently the final uncertainty in the frictional coefficient.
Regardless, the Auto Fit gives a slope of $0.1782 \approx 0.18$N, with max/min values of
$\approx 0.21$N and $0.16$N, respectively.
This means an uncertainty of $\frac{0.21-0.16}{2}\approx0.03$N in absolute terms, or a 13\% relative uncertainty.

Since the origin lies above the line of maximum slope, but below the line of minimum slope,
no major systematic bias is perceived.

This value can be compared with those obtained in Murase's study, mentioned in \textbf{Section 4}.
Their coefficient for stainless steel ($\approx1.2$) falls outside of the range given by our uncertainties, but
the coefficient for mild steel ($\approx 1.6$) is within the range.
This deviation is expected and potential causes of it were explained in \textbf{Section 4}.
Additionally, if \textbf{Figure 2} of the box and its railings are observed closely, we can see that they're
slightly rusty, so the metal in question is probably not stainless steel.

\section{Conclusions}
I am amazed by the lack of availability of existing data regarding this coefficient measured in my investigation.
Perhaps it is simply not relevant in engineering, to have steel slide against wood?
The majority of data on steel's dynamic frictional coefficients I was able to find online
had it slide against rubber, steel, copper, ice, plasic, or concrete, and often included lubricant oil or water.

I think Murase's study should be performed and documented again; the format is slightly outdated and the results cannot really be compared to anything else readily available.
\newpage
\printbibliography

\end{document}