\documentclass[a4paper, 12pt]{article}

%Paragraph jumps and indentation
\setlength{\parskip}{1.6em}
\setlength{\parindent}{1cm}

%Border
\usepackage[left=1in, right=1in, top=1in, bottom=1in]{geometry}

%Double spacing
\usepackage{setspace}
\doublespacing
%Packages
\usepackage{amsmath}
\usepackage[dvipsnames]{xcolor}
\usepackage{mathtools}
\usepackage{amsfonts}
\usepackage{titlesec}
\usepackage{mhchem}

%Images
\usepackage{graphicx}
\graphicspath{ {./images/} }
\usepackage{wrapfig}
\usepackage{float}

%Tables
\usepackage{multirow}
\usepackage{array}
\usepackage{tabu}
\titleformat{\section}
{\normalfont\large\bfseries}{\thesection}{1em}{}
\titleformat{\subsection}
{\normalfont\large\bfseries}{\thesubsection}{1em}{}

%Equation numbering
\counterwithin{equation}{section}

%links
\usepackage{hyperref}
\hypersetup{
    colorlinks=true,
    linkcolor=blue,
    filecolor=magenta,      
    urlcolor=cyan,
    pdftitle={Overleaf Example},
    pdfpagemode=FullScreen,
    }
\urlstyle{same}

%Diagrams
\usepackage{pgfplots}
\pgfplotsset{compat=newest}
\usetikzlibrary{positioning, arrows.meta}
\usepgfplotslibrary{fillbetween}

%Theorems
\newtheorem{theorem}{Theorem}[section]
\newtheorem{definition}{Definition}[section]
\newtheorem{lemma}{Lemma}[section]
\newtheorem{corollary}{Corollary}[section]

%Bibliography
\usepackage[backend=biber]{biblatex}
\addbibresource{refs.bib}

\newcommand{\diff}{\frac{\mathrm{d}}{\mathrm{d}t}}



\begin{document}

\begin{titlepage}
  \begin{center}
    Internal Assessment: Physics\\
    \vspace*{4cm}
    {\Large \textbf{Computing the dynamic frictional coefficient $\mu_d$ between wood and steel by measuring the relationship between the force of friction and mass.}}\\
    
    \vspace{1cm}
    \textbf{Research Question:}
    How does changing the mass of the box sliding against a plank affect the force of friction it experiences?
    \vfill
    \today\\
    Word count: 420
  \end{center}
\end{titlepage}

\section{Introduction}
Idk man
\section{Background Information}
Theory says dybamic friction is proportional to the normal force.
This means that a frictional coefficient exists, which is $F_\mu/N$ where $N$ is the normal force.
\subsection{Relevant Theory}
\begin{quote}
  \hspace{-1em}\vrule width 1pt\hspace{1em}%
  \begin{minipage}[t]{0.95\linewidth}
  ``The rate limiting reaction is believed to be the initial reaction between hydrogen peroxide and the iron oxides \cite{Miller1995,ValentineWang1998,Miller1999}.
  Additionally, the results reveal that the decomposition of \ce{H2O2} follows pseudo-first order kinetics:
  \begin{equation}
    -\dfrac{\mathrm{d}[\ce{H2O2}]}{\mathrm{d}t} = k_\mathrm{app}[\ce{H2O2}]
  \end{equation}
  and thus:
  \begin{equation}
    \ln\left(\dfrac{[\ce{H2O2}]}{[\ce{H2O2}]_0}\right) = -k_\mathrm{app}t
  \end{equation}
  where $k_\mathrm{app}$ is the apparent first order rate constant,
  and $[\ce{H2O2}]$ and $[\ce{H2O2}]_0$ are the concentrations of \ce{H2O2} in the solution at any time $t$ and time zero, respectively.\cite{Huang}''
  \end{minipage}
\end{quote}

\noindent The data presented in \textbf{Section 4} agrees with theory, that the reaction is pseudo-first order.
The aim of this investigation is to measure how $k_\mathrm{app}$ changes in relation to the amount of catalyst used.

Now a general expression for pressure as a function of time  will be mathematically derived from (2.2).
\begin{center}
\begin{tabular}{|c >{\raggedright\arraybackslash}p{5cm}|c >{\raggedright\arraybackslash}p{5cm}|}
  \hline
  \multicolumn{4}{|c|}{Notation} \\
  \hline
  $n(\ce{H2O2})$&\ce{H2O2} in solution at time $t$&$P$&Pressure at time $t$\\
  $n(\ce{O2})$&\ce{O2} released up to $t$ &$P_0$&Initial pressure\\
  $n_0$&Initial \ce{H2O2} in solution&$P_\infty$&Final pressure\\
  $n_\mathrm{air}$&Initial air molecules&$V_l$&Solution volume\\
  $R$&Ideal gas constant&$V_f$&Flask volume\\
  $T$&Temperature&$V_g$&$= V_f-V_l$, Gas volume\\
  \hline
\end{tabular}
\end{center}

\noindent The catalysed decomposition is
\[
\ce{2H2O2 (aq) ->[MnO2] 2H2O (l) + O2 (g)}
\]
From stoichiometry,
\[
n(\ce{O2})=\frac{n_0-n(\ce{H2O2})}{2}
\quad\Longrightarrow\quad
n(\ce{H2O2})=n_0-2\,n(\ce{O2}).
\]
With $[\ce{H2O2}]=n(\ce{H2O2})/V_l$ and $PV_g=(n_{\mathrm{air}}+n(\ce{O2}))RT$,
\[
n(\ce{O2})=\frac{PV_g}{RT}-n_{\mathrm{air}}
\]
so
\[
[\ce{H2O2}]
=\frac{n_0+2n_{\mathrm{air}}-2PV_g/(RT)}{V_l}
\]
At completion $n(\ce{H2O2})=0$, hence $n_0+2n_{\mathrm{air}}=2P_\infty V_g/(RT)$ and therefore
\[
[\ce{H2O2}]=\frac{2(P_\infty-P)\,V_g}{RT\,V_l}
\]
Combining this with the integrated pseudo–first order law
\[
-k_{\mathrm{app}}t=\ln\frac{[\ce{H2O2}]}{[\ce{H2O2}]_0},\qquad [\ce{H2O2}]_0=\frac{n_0}{V_l}
\]
and simplifying yields
\[
2(P_\infty-P)\frac{V_g}{RT}=n_0 e^{-k_{\mathrm{app}}t}
\]
and hence
\[
P(t)=P_\infty-\frac{n_0RT}{2V_g}\,e^{-k_{\mathrm{app}}t}
\]
For brevity define
\[
c:=\frac{n_0RT}{2V_g},
\]
giving the general form
\[
\boxed{\,P(t)=P_\infty-c\,e^{-k_{\mathrm{app}}t}\,}.
\]

\subsection{Existing Results}

For pyrolusite (the primary manganese ore) powder, the highest reported specific rate constant is 
\[
k_\mathrm{pyro} = 0.061 \,\mathrm{min^{-1}\,mM^{-1}},
\]
measured at room temperature in 200 mL of solution containing $29.4$ mM \ce{H2O2}.\cite{Do}
This suggests that the rate constant is proportional to the concentration of \ce{MnO2} in solution.

The results obtained in this investigation will be compared with this result.
\textbf{Hypothesis.} Ideally a proportional relationship between the mass of \ce{MnO2} and $k_\mathrm{app}$ is observed in this investigation, and the result does not deviate majorly from the above value.

\section{Experimental}
\subsection{Materials \& Apparatus}
\begin{tabular}{>{\raggedright\arraybackslash}p{0.5\textwidth} >{\raggedright\arraybackslash}p{0.5\textwidth}}
  \multicolumn{2}{c}{\textbf{Materials}} \\ \hline
  $33\%$ (v/v) \ce{H2O2} (aq), diluted to $3.3\%$ & $10 \pm 0.5$ mL per trial \\
  Pure distilled water & $20 \pm 0.5$ mL per trial \\
  Fine \ce{MnO2} powder & $\approx 1$g total \\ \hline
  \multicolumn{2}{c}{\textbf{Equipment}} \\ \hline
  220 mL Erlenmeyer flask&Thermometer\\
  20 mL Medical syringe ($\pm 0.5$ mL)&Electronic scale ($\pm 0.001$ g)\\
  Vernier Pressure sensor\footnote{\url{https://www.vernier.com/product/gas-pressure-sensor/}, Accessed \today}&Rubber stopper\\
  Magnetic stir bar and stir plate&Parafilm \\\hline
\end{tabular}


\noindent This fit gives the specific rate constant $k_\mathrm{spec}\approx 0.00009093\ s^{-1}\ \mathrm{mM}^{-1} = 0.0054558\ \mathrm{min}^{-1}\ \mathrm{mM}^{-1}$,
which is within an acceptable range of the known $k_\mathrm{pyro} = 0.061\ \mathrm{min}^{-1}\ \mathrm{mM}^{-1}$ from \textbf{Section 2.2}.
\subsection{Conclusion}

The data support a direct, near-proportional increase of the apparent first-order rate constant $k_{\mathrm{app}}$ with suspended \ce{MnO2} concentration.
This matches the general expectation that more active surface yields faster reaction, but the measured specific rate is lower than literature ($0.061\ \mathrm{min^{-1}\,mM^{-1}}$); plausible causes include incomplete \ce{O2} capture (leakage, connector losses), variable effective surface area (agglomeration or settling of powder), the initial connection artifact, and instrument limits (pressure sensor accuracy, uncertain $V_g$).
These uncertainties (notably $V_l \pm 3.3\%$ and sensor $\pm 2$ kPa) moderately affect $k_{\mathrm{app}}$ and explain observed scatter and the anomalously low 0.05 g point, so quantitative agreement with literature is tentative though the trend is robust.

Methodological limitations with meaningful impact:
reliance on headspace pressure rather than direct concentration assays, imperfect seals, limited replicates per mass and a narrow mass range, and possible temperature/stirring inconsistencies.
Strengths include a clear theoretical derivation linking $P(t)$ to $[\ce{H2O2}]$ and continuous high-frequency data collection.

Realistic, relevant improvements: use pressure-rated fittings or water-displacement gas collection; immobilize catalyst or better disperse powder;
increase replicate number and mass range; directly assay $[\ce{H2O2}]$ (titration or spectrophotometry) for cross-validation;
calibrate sensor and measure $V_g$ precisely.
With these changes the proportional relationship can be tested more reliably and quantitative agreement improved.


\newpage

\printbibliography

\end{document}