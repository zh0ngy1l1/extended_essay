\documentclass[a4paper, 12pt]{article}

%Paragraph jumps and indentation
\setlength{\parskip}{1.6em}
\setlength{\parindent}{0.618cm}

%Border
\usepackage[left=1in, right=1in, top=1in, bottom=1in]{geometry}

%Double spacing
\usepackage{setspace}
\doublespacing

%Packages
\usepackage{amsmath}
\usepackage[dvipsnames]{xcolor}
\usepackage{mathtools}
\usepackage{amsfonts}
\usepackage{amssymb}
\usepackage{titlesec}
\usepackage{enumitem}
\usepackage{cancel}

%Images
\usepackage{graphicx}
\graphicspath{ {./images/} }
\usepackage{wrapfig}
\usepackage{float}

%Tables
\usepackage{multirow}
\usepackage{array}
\usepackage{tabu}
\titleformat{\section}
{\normalfont\large\bfseries}{\thesection}{1em}{}
\titleformat{\subsection}
{\normalfont\large\bfseries}{\thesubsection}{1em}{}
\usepackage{tabularx} 
\usepackage{ragged2e}

%Equation numbering
\counterwithin{equation}{section}
\usepackage{hyperref}
\urlstyle{same}

%Diagrams
\usepackage{pgfplots}
\pgfplotsset{compat=newest}
\usetikzlibrary{positioning, arrows.meta}
\usepgfplotslibrary{fillbetween}

%Theorems
\usepackage{amsthm}
\newtheorem{theorem}{Theorem}
\newtheorem{lemma}{Lemma}
\newtheorem{corollary}{Corollary}
\newtheorem{question}{Question}

\theoremstyle{definition}
\newtheorem{definition}{Definition}
\newtheorem{example}{Example}
\newtheorem{problem}{Problem}
\newtheorem*{terminology}{Terminology}
\newtheorem*{notation}{Notation}

%Bibliography
\usepackage[
  backend=biber
]{biblatex}
\addbibresource{refs.bib}

%Fancy footnote
\renewcommand{\thefootnote}{\fnsymbol{footnote}}

\begin{document}

\begin{titlepage}
  \begin{center}
    Subject: Mathematics\\
    \vspace*{4cm}
    {\Large \textbf{Investigating patterns in the product of Euler's totient function $\varphi$ and the sum of divisors function $\sigma$}}\\

    \vspace{1cm}
    \textbf{Research Question:}
    What gives rise to the patterns in the graph of $n^{-2}\varphi(n)\sigma(n)$?\\

    \vfill
    Word count: 1729
  \end{center}
\end{titlepage}

\tableofcontents

\newpage

\section{Introduction}
Write this when most of the essay is finished
%As the British mathematician G. H. Hardy most eloquently argues in his essay \textit{A Mathematician's Apology} (1940)\cite{hardy}, pure mathematics is pursued for its aesthetic value despite having no practical applications.
%When encounting a result that seems plausible but is not formally proven, a mathematician cannot help being compelled to seek for a formal justification.
%The investigations done in the production of this paper are motivated by precisely this: It is a classic textbook example to show the bounds of $$

%Explanation why this topic is relevant and important to a student.
%Clearly stated the research question
%Methodology of the research (future tense)

\begin{figure}[h]
  \centering
  \includegraphics[width=\linewidth]{2million_simple.png}
  \caption{what?}
\end{figure}

\newpage
\section{Prerequisite Knowledge, Notation, and Definitions}
The inarguably most important theorem in number theory is the following:
\begin{theorem}[Fundamental Theorem of Arithmetic]
  Every positive integer has an unique prime factorization.\cite{apostol}
\end{theorem}
This theorem is crucial in our ensuing discussion, because it allows us to express any integer $n\geq 2$ uniquely in terms of its $k$ prime divisors $p_1, p_2, ..., p_k$,
along with the positive integers $\alpha_1, \alpha_2, ..., \alpha_k$ which indicate the number of times that $n$ is divisible by the corresponding prime factor.
With this, we obtain the familiar form

\begin{equation}
  n = p_1^{\alpha_1} p_2^{\alpha_2} \dots p_k^{\alpha_k} = \prod_{i=1}^{k}p_i^{\alpha_i}
\end{equation}

In the special case $n=1$, there are no prime factors, and by convention
\begin{equation}
  \textit{the empty product equals 1.}
\end{equation}

The reader should be comfortable with $\Pi$-notation for sequential products, which is similar to $\Sigma$ but for multiplication.

\subsection{Introduction to Sets}
A \textit{set} is an \textit{unordered} collection of elements.
The discussion in this paper only involves sets of numbers, such as $\mathbb{Z}$, used to denote the set of integers,
and $\mathbb{R}$, used to denote the set of real numbers.

Here \textit{unordered} means that for example, the set $A =\{1, 2\}$ is the same as
the set $B=\{2, 1\}$.
When two sets $A, B$ are the same, we write $A = B$.

\begin{notation}
  \begin{itemize}
    Let $A$, $B$, and $S$ be sets.
    \item The notation $x \in S$ means that $x$ is an element of the set $S$.
    \item For a finite set $S$, the symbol $|S|$ denotes the number of elements contained in $S$.
    \item The union $A \cup B$ is the set of all elements that belong to $A$, to $B$, or to both.
    \item The intersection $A \cap B$ is the set of all elements that belong to both $A$ and $B$.
    \item The difference $A \setminus B$ is the set of all elements that belong to both $A$ but not $B$.
  \end{itemize}
\end{notation}\vspace{0.5em}

\begin{example}\text{}

  The set $\{1,2,4,7,8,11,13,14\}$ has size $\left|\{1,2,4,7,8,11,13,14\}\right|=4$.

  For sets $A= \{2,4,6,8,10,12\},\ B=\{3,6,9,12\}$,
  their union is $A\cup B = \{2, 3, 4, 6, 8, 9, 10, 12\}$,
  and their intersection is $A\cap B = \{6, 12\}$.
  The difference $A\setminus B$ is $\{2,4,8,10\}$.
\end{example}

\subsection{Set-builder Notation}
Set-builder notation is a way for mathematicians to define sets symbolically rather than verbosely.
Basic set-builder notation is used frequently throughout this paper.

\begin{notation}
  The set $S$ of elements in the set $A$ that fulfill a certain condition is defined, i.e. `built', with the following notation:
  \[S = \{a \in A\ :\ \text{conditions for }a\}\]
\end{notation}

In this case, $S$ is a subset of $A$, because every element of $S$ is an element of $A$.
This is denoted by $S\subseteq A$.
The reader should understand the following examples before proceeding:

\begin{example}
  Build the set of positive integers, denoted $\mathbb{Z}^+$, by choosing the integers which are positive:
  \begin{equation}
    \mathbb{Z}^+  = \{x \in \mathbb{Z}\ :\ x > 0 \}
  \end{equation}
\end{example}\vspace{0.5em}


\begin{example}
  Build the set of positive divisors of an integer $n\neq 0$, denoted $\mathcal{D}_n$, by choosing the positive integers that divide $n$:
  \begin{equation}
    \mathcal{D}_n = \{x \in \mathbb{Z}^+\ :\  x \mid n \}
  \end{equation}
  The notation $a\mid b$ means $a$ divides $b$.
\end{example}\vspace{0.5em}

\begin{example}
  Recall that prime numbers are positive integers greater than 1 whose only positive divisors are 1 and itself.
  Build the set of primes, denoted $\mathbb{P}$, using the set of positive divisors defined in (2.4):
  \begin{equation}
    \mathbb{P} = \left\{x \in \mathbb{Z}^+\ :\ x>1,\ \mathcal{D}_x = \{1,\ x\} \right\}
  \end{equation}
  In this case, multiple conditions on $x$ are necessary, and notationally we separate them with a comma.  
\end{example}

More complicated sequential sums and products also involve require set-builder-like notation in the subscripts, used to impose restrictions on the what to take the sum or product over.
(2.6) in \textbf{Definition 1} serves as an example of this.

\subsection{Key Definitions}
The following functions, defined for any positive integer $n$, are the central focus of this study.

\begin{definition}[Sum of Divisors Function $\sigma$]
  Define $\sigma(n)$ as the sum of all positive divisors of $n$.

  Mimicking set-builder notation, write
  \begin{equation}
    \sigma(n) = \sum_{d \in \mathcal{D}_n} d
  \end{equation}
\end{definition}

\begin{example}
  Compute $\sigma(12)$:
  Since the positive divisors of 12 are $\mathcal{D}_{12}=\{1,\ 2,\ 3,\ 4,\ 6,\ 12\}$,
  taking their sum yields $\sigma(15)=1+2+3+4+6+12=28$
\end{example}

\vspace{0.5em}\begin{definition}[Euler's Totient Function $\varphi$]
  Define $\varphi(n)$ as the number of positive integers less than or equal to $n$ and \textit{coprime} to $n$.
  Formally, define
  \begin{equation}
    \varphi(n) = \left|\{x\in \mathbb{Z}^+\ :\ x\leq n,\ \gcd(x,n)=1\}\right|
  \end{equation}
\end{definition}
\begin{terminology}
  We say that `$a$ is coprime to $b$', or `$a$ and $b$ are coprime', if $\gcd(a,b)=1$.
\end{terminology}

\begin{example}
  Compute $\varphi(12)$:

  \noindent The positive integers less than or equal to 15 and coprime to 15 are 1, 5, 7, 11.
  Since there are 4 such numbers, $\varphi(12)=4$.
\end{example}

\newpage

\section{The Problem of Study}
This investigation was inspired by the following problem from the undergraduate textbook \textit{Introduction to Analytic Number Theory} \cite{apostol}.

\begin{problem}
  Show that $\displaystyle \frac{6}{\pi^2}<\frac{\varphi(n)\sigma(n)}{n^2}<1$ for all $ n\geq 2.$
\end{problem}

\begin{figure}[h]
  \centering
  \includegraphics[width=0.8\textwidth]{2500_bounds.png}
  \caption{Graph for \textbf{Problem 1}}
\end{figure}

\textbf{Figure 2} visualizes the scenario by plotting points $(n, \frac{\varphi(n)\sigma(n)}{n^2})$ (in blue) and the lower-- and upper bounds (in red) on the Cartesian plane.
Then, the task in \textbf{Problem 1} is to show that all blue points fall in between the red lines.

The research question of this investigation, regarding `patterns' in the graph can now be put concretely.
In addition to solving \textbf{Problem 1}, this paper will develop the necessary tools to answer the following questions.

\begin{question}
  Why are `denser lines' formed?
\end{question}
This question regards the horizontal lines that many of the points seem to fall in, at $y=1,\ y=0.75$, and $y=0.89$ for example.
This is one of the first questions arising upon seeing \textbf{Figure 2}, since the orderly patterns strike as unexpected in such a scattered and seemingly random distribution.
After addressing \textbf{Question 1}, the next natural question is,

\begin{question}
  Where can `denser lines' form?
\end{question}
This is also a natural thing to ask.
We see many dense lines between 0.95 and 1, and much fewer below 0.70.

\section{Solving Problem 1}
First, we must improve our undrstanding on the functions $\sigma$ and $\varphi$. 
The following product forms for these functions in terms of the prime factorization of $n$ can be derived from their definitions.

\begin{lemma}
  For any positive integer $n$ with prime factorization $n = \prod_{i=1}^{k} p_i^{\alpha_i}$\\
  \begin{align}
    \sigma(n)  &= \prod_{i=1}^{k} \dfrac{p_i^{\alpha_i+1}-1}{p_i-1}\\
    \varphi(n) &= \prod_{i=1}^{k} p_i^{\alpha_i-1}(p_i-1)
  \end{align}
\end{lemma}
Sections (4.1) and (4.2) are proofs for these two statements.
\subsection{Deriving the Product Form of $\sigma$}
To do this, we first show that $\sigma$ is \textit{multiplicative}, which means
\begin{equation}
  \sigma(mn)=\sigma(m)\sigma(n)
\end{equation}
holds for all coprime positive integers $m, n$.

Let $d$ be a positive divisor of $mn$, and write the prime factorization of $d$:
\begin{equation}
  d=\prod_{i=1}^{k}p_i^{\alpha_i}
\end{equation}
In this form, each term $p_i^{\alpha_i}$ either 
\vspace{-0.5em}\begin{enumerate}
  \item divides $m$ and is coprime to $n$, or
  \item divides $n$ and is coprime to $m$,
\end{enumerate}\vspace{-0.5em}
because $d$ is a divisor of $mn$ and $m$, $n$ are coprime.
That means we can (uniquely) write
\begin{equation}
  d=d_m d_n
\end{equation}
where $d_m$ is the product of the terms which correspond
to case 1., and $d_n$ is the product of the terms
which correspond to case 2.

Conversely, if $d_m$ and $d_n$ are positive divisors
of $m$ and $n$, respectively, then $d_m d_n$ is a
divisor of $mn$. This means that 
\begin{lemma}
  There is a \textit{one-to-one correspondence}
  between elements in $\mathcal{D}_{mn}$ and pairs of
  elements in $\mathcal{D}_m$ and $\mathcal{D}_n$,
  for coprime positive integers $m, n$.
\end{lemma}

\begin{example}
  Let $m=9,\ n=14$.
  The divisors of $9$ are $\mathcal{D}_9=\{1,\ 3,\ 9\}$ and of $14$ are
  $\mathcal{D}_{14}=\{1,\ 2,\ 7,\ 14\}$.
  Every divisor of $mn=126$ is a product of one divisor
  of $9$ and one divisor of $14$:
  \[
  \begin{aligned}
    1  &= 1\cdot1,  & 2  &= 1\cdot2,  & 7  &= 1\cdot7,  & 14  &= 1\cdot14,\\
    3  &= 3\cdot1,  & 6  &= 3\cdot2,  & 21 &= 3\cdot7,  & 42  &= 3\cdot14,\\
    9  &= 9\cdot1,  & 18 &= 9\cdot2,  & 63 &= 9\cdot7,  & 126 &= 9\cdot14.
  \end{aligned}
  \]
  Thus $\mathcal D_{126}=\{1,\ 2,\ 3,\ 6,\ 7,\ 9,\ 14,\ 18,\ 21,\ 42,\ 63,\ 126\}$.
\end{example}
\noindent Thanks to \textbf{Lemma 2}, \textit{multiplicativity} can be established using \textbf{Definition 1}.
\begin{align}
\sigma(mn) =&\sum_{d\in \mathcal{D}_{mn}} d\\
=&\sum_{d_m \in \mathcal{D}_m}\sum_{d_n \in \mathcal{D}_n} d_m d_n\\
=&\sum_{d_m \in \mathcal{D}_m} d_m \left(\sum_{d_n \in \mathcal{D}_n} d_n\right)\\
=&\left(\sum_{d_m \in \mathcal{D}_m} d_m\right)\left(\sum_{d_n \in \mathcal{D}_n} d_n\right)=\sigma(m)\sigma(n).
\end{align}
\vspace{-0.5em}\begin{enumerate}[label=(4.\arabic*), start=6]
  \item This is \textbf{Definition 1}.
  \item By \textbf{Lemma 2}, divisors of $mn$ correspond uniquely to products $d_m d_n$ with $d_m\mid m$, $d_n\mid n$, so the sum may be rewritten as a double sum over such pairs.
  \item The inner and outer sums are independent, allowing the factors $d_m$ and $d_n$ to be separated.
  \item Since the inner sum depends only on $n$, it can be factored out of the outer sum.
\end{enumerate}
This lets us express $\sigma(n)$ in terms of the prime factorization
$n=\prod_{i=1}^{k}p_i^{\alpha_i}$ for any positive integer $n$.
Since $p_1^{\alpha_1},\ p_2^{\alpha_2},\ \dots,\ p_k^{\alpha_k}$ are all coprime,
\textit{multiplicativity} lets us separate the terms $p_1^{\alpha_1},\ p_2^{\alpha_2},\ \dots,\ p_k^{\alpha_k}$
out, one by one, to finally reach the form in (4.15):
\begin{align}
  \sigma(n) = &\sigma(p_1^{\alpha_1}p_2^{\alpha_2} \dots p_k^{\alpha_k})\\
  = &\sigma(p_1^{\alpha_1}) \cdot \sigma(p_2^{\alpha_2} \dots p_k^{\alpha_k})\\
  = &\sigma(p_1^{\alpha_1}) \cdot \sigma(p_2^{\alpha_2}) \cdot \sigma(p_3^{\alpha_3} \dots p_k^{\alpha_k})\\
  = &\dots\\
  = &\sigma(p_1^{\alpha_1}) \cdot \sigma(p_2^{\alpha_2}) \cdot \dots \cdot \sigma(p_k^{\alpha_k})\\
  = &\prod_{i=1}^{k}\sigma(p_i^{\alpha_i})
\end{align}
To finish, we only need to compute $\sigma(p^\alpha)$ when $p$ is prime and $\alpha$ is any positive integer.
This is easy since
\begin{equation}
  \mathcal{D}_{p^\alpha}=\{1,\ p,\ p^2,\ \dots,\ p^\alpha\}
\end{equation}
which implies $\sigma(p^\alpha)$ is simply a geometric series
\begin{equation}
  \sigma(p^{\alpha})=1+p+p^2+\dots+p^\alpha = \frac{p^{\alpha+1}-1}{p-1}
\end{equation}
Finally, substitute this into (4.15) to finish:
\[
  \sigma(n) = \prod_{i=1}^{k} \frac{p_i^{\alpha_i+1}-1}{p_i-1} \tag{4.1}
\]
This is true for all $n>1$.


\begin{example}
  Compute $\sigma(2025)$.
  \begin{align*}
    \sigma(2025) = &\sigma(3^4\cdot 5^2)\\
    = &\sigma(3^4)\cdot\sigma(5^2)\\
    = &\frac{3^5-1}{3-1} \frac{5^3-1}{5-1}\\
    = &121 \cdot 31 = 3751
  \end{align*}
\end{example}

\subsection{Deriving the Product Form of $\varphi$}
Let $n$ be a positive integer with prime factorization
\begin{equation}
  n=\prod_{i=1}^{k} p_i^{\alpha_i} 
\end{equation}

Recall the definition of $\varphi$ from \textbf{Definition 2}:
\begin{equation}
  \varphi(n) = \left|\{x\in \mathbb{Z}^+\ :\ x\leq n,\ \gcd(x,n)=1\}\right|
  \tag{2.7}
\end{equation}
In other words, we can compute $\varphi(n)$ using the set of positive integers less than
or equal to $n$, by removing those which are a multiple of some prime factor of $n$, and
counting how many remains.
This works because those which are erased are not coprime to $n$, and all the remaining ones are.
\begin{example}
  Suppose $n=15 = 5 \cdot 3$, and compute $\varphi(15)$:

  The multiples of 3 less than 15 are
  $\{3,\ 6,\ 9,\ 12,\ 15\}$,
  and the multiples of 5 less than 15 are
  $\{5,\ 10,\ 15\}$. If we remove these from the positive integers less than or equal to 15,
  the integers that remain are
  \begin{equation}
    \{1,2,\not3,4,\not5,\not6,7,8,\not9,\not10,11,\not12,13,14,\not15\}
    =\{1,\ 2,\ 4,\ 7,\ 8,\ 11,\ 13,\ 14\}
  \end{equation}
  so $\varphi(15) = 8$
\end{example}
The line of thought in \textbf{Example 9} motivates the following definition:
\begin{definition}
  For some prime divisor $p\mid n$, let $S_p (n)$ denote the set of multiples of $p$ which are less than or equal to $n$.
  Formally, define the set
  \begin{equation}
    S_p (n) = \left\{x\in\textbf{Z}^+\ :\ x\leq n,\ p \mid x\right\}
  \end{equation}
\end{definition}
Notice how $\{3,\ 6,\ 9,\ 12,\ 15\} = S_3 (15)$ and $\{5,\ 10,\ 15\}= S_5 (15)$ in \textbf{Example 9}.

To generalize the idea in the example, we should consider the following set:
\begin{equation}
  S_{p_1}(n)\cup S_{p_2}(n)\cup \dots \cup S_{p_k}(n)
\end{equation}
This is the set we remove from $\{1,\ 2,\ \dots,\ n\}$, so that the remaining numbers
are all coprime to $n$ and hence make up $\varphi(n)$.

Actually, size of the set in (4.21) is precisely $n-\varphi(n)$.

prove inclusion-exclusion (pp. 231 - 233).




\newpage
\text{\huge \color{red}\textbackslash section\{ROUGH DRAFT TERRITORY\}}
\begin{figure}[h]
  \centering
  \begin{minipage}{0.49\linewidth}
    \centering
    \includegraphics[width=\linewidth]{2500_simple.png}
    \caption{Plotting the Sequence}
    \label{fig:simple}
  \end{minipage}
  \hfill
  \begin{minipage}{0.49\linewidth}
    \centering
    \includegraphics[width=\linewidth]{2500_minmax.png}
    \caption{Highlighting Minima \& Maxima}
    \label{fig:minmax}
  \end{minipage}
\end{figure}

After discussing these regions with many points, we proceed to discuss the opposite:
Are there any regions with no points?
Formally, we may ask,
\begin{question}
  Does there exist a nonempty open interval
  $I\subset\left(\frac{6}{\pi^2},1\right)$
  such that 
  $a_n \notin I$ holds for all $ n\geq 2$?
\end{question}

Notice how this formulation looks familiar?
Similarly to \textbf{Question 1}, the answer here is also \textbf{No.}
This is actually a generalization of that question:
Here we are showing that the sequence $a_n$ gets arbitrarily close to \textit{any} point in the interval $\left(\frac{6}{\pi^2},1\right)$,
while \textbf{Question 1} only required us to show that the sequence gets arbitrarily close to its boundary.

\newpage
Then show that the basel problem $\infty$ sum is the $\infty$ product: Page 230 (pdf: 242) of \cite{apostol}.
Then solve \textbf{Problem.} by writing $\dfrac{\varphi(n)\sigma(n)}{n^2}$ in terms of primes:
\[\frac{\varphi(n)\sigma(n)}{n^2} = \prod_{i=1}^{k} \dfrac{p_i^{\alpha_i+1}-1}{p_i^{\alpha_i+1}}\]
(we get this by plugging Lemma 1)

To finish the problem, see the maximum of each term in this product, and the minimum of each term.
Each term is at least $\dfrac{p_k^2-1}{p_k^2}$ and at most arbitrarily close to 1 but less than $1$.

Then the global maximum is strictly below 1 (we only consider $n\geq2$\footnote{maybe consider $n=1$ too but that's a matter of my taste \& preference})
Then the global minimum is strictly above $\prod_{p\in \mathbb{P}}\dfrac{p^2-1}{p^2} = \dfrac{6}{\pi^2}$ (Basel Problem)

So \textbf{Problem 1.} is done.

\section{Questions}
Next answer Question 1. Also rewrite it not using the interval because it's IB AA HL, not Topology 101.

The answer is: no, the bounds can't be uniformly improved.
For upper bound this is trivial (n = $2^k$ for very big $k$)

For lower bound it's what convergence means. Now that I think about it, Question 1 isn't really worth doing since it's so fricking easy.

Question 2.
The topmost `dense line' is primes, (or prime powers? That would make the discussion more complete but also longer and not that much more impressive)
\begin{definition}[`dense line' should I come up with a better name :(]
  We define a subsequence $b_n$ of the sequence $a_n$ to be a `dense line' if it
\begin{itemize}
  \item increases
  \item approaches some value $t$
  \item $b_n$ = $a_g(n)$ and $g(n) < c x \ln x$\footnote{Here $x \ln x$ is the asymptotic growth of primes (also the asymptotic growth of prime powers, cool! This condition says `$b_n$ must be dense enough.')}
\end{itemize}
\end{definition}


Actually, we get a dense line $b_n = a_{c p_n}$ for any positive integer d, and the line is denser the smaller c is (obviously). here $p_n$ is the $n$th smallest prime.
Though the issue is that if I define `dense lines' like this, it's hard to verify every dense line is generated by this description.
In fact, I don't even think that's true... Maybe it is? But how do I define it better? This is the only formal definition I could come up with!

But that's enough knowledge about the dense lines that we can graph them! Is this a satisfying enough answer to Question 2?
\begin{figure}[h]
  \centering
  \begin{minipage}{0.49\linewidth}
    \centering
    \includegraphics[width=\linewidth]{2500_densest.png}
    \caption{First 6 `dense lines'}
    \label{fig:densest}
  \end{minipage}
  \hfill
  \begin{minipage}{0.49\linewidth}
    \centering
    \includegraphics[width=\linewidth]{2million_simple.png}
    \caption{plotting up to $n=2,000,000$}
    \label{fig:2million}
  \end{minipage}
\end{figure}

Now question 3. Look at figure 4. We see more blue `dense lines' than previously, but also white regions with very few points.
We are motivated to ask: Do dense lines cover everything eventually? Will there be some strip with no points whatsoever?
The answer is no, and we present an algorithm which shows this. It's really cool and relies on Nagura's bound:
\begin{theorem}
  For all $n\geq 25$ there is always a prime $p$ with $n < p < 1.2n$
\end{theorem}

And a bit of computer brute-force for $3 < n < 25$

and a tiny bit of brute-force by hand for $n=2, n=3$

After that though Nagura + induction can get the remaining cases.
I think this is my coolest result, and I only recently come up with it




\section{Concluding Remarks}
I don't know what to say here.
It's been fun investigating this, but seeing that the lines were just primes and constant multiples of primes is kinda anticlimactic for me; I like very hard problems and felt much more thrilled solving \textbf{Question 3.}

Also I need to write an introduction\dots

\newpage

\printbibliography

\end{document}