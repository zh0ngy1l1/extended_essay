\documentclass[a4paper, 12pt]{article}

%Paragraph jumps and indentation
\setlength{\parskip}{1.6em}
\setlength{\parindent}{1.25cm}

%Border
\usepackage[left=1in, right=1in, top=1in, bottom=1in]{geometry}

%Double spacing
\usepackage{setspace}
\doublespacing

%Packages
\usepackage{amsmath}
\usepackage[dvipsnames]{xcolor}
\usepackage{mathtools}
\usepackage{amsfonts}
\usepackage{titlesec}

%Images
\usepackage{graphicx}
\graphicspath{ {./images/} }
\usepackage{wrapfig}
\usepackage{float}

%Tables
\usepackage{multirow}
\usepackage{array}
\usepackage{tabu}
\titleformat{\section}
{\normalfont\large\bfseries}{\thesection}{1em}{}
\titleformat{\subsection}
{\normalfont\large\bfseries}{\thesubsection}{1em}{}

%Equation numbering
\counterwithin{equation}{section}
\usepackage{hyperref}
\urlstyle{same}

%Diagrams
\usepackage{pgfplots}
\pgfplotsset{compat=newest}
\usetikzlibrary{positioning, arrows.meta}
\usepgfplotslibrary{fillbetween}

%Theorems
\newtheorem{theorem}{Theorem}
\newtheorem{definition}{Definition}

%Bibliography
\usepackage[
  backend=biber
]{biblatex}
\addbibresource{refs.bib}

\begin{document}

\begin{titlepage}
  \begin{center}
    Subject: Mathematics\\
    \vspace*{4cm}
    {\Large \textbf{Investigating patterns in the product of Euler's totient function $\varphi$ and the sum of divisors function $\sigma$}}\\

    \vspace{1cm}
    \textbf{Research Question:}
    What gives rise to the patterns and bounds in the graph of $y=\varphi(x) \cdot \sigma(x)$, and how does this relate to the Riemann zeta function and the Basel problem?\\

    \vspace{4cm}
    Word count: 225
    \vfill
    \vspace{0.1cm}
    Zhongyi Li, M26
  \end{center}
\end{titlepage}

\section{Abstract}
This paper aims to investigate patterns in the graph of $y=\varphi(x)\cdot\sigma(x)$.
\newpage


\tableofcontents

\newpage

\section{Introduction}
Broadly speaking, number theory is the branch of mathematics concerned with the properties of integers.
In a school setting, this often includes topics such as divisibility rules, prime factorization, and, most importantly, the \textbf{Fundamental Theorem of Arithmetic}.

\begin{theorem}[Fundamental Theorem of Arithmetic]
  Every integer $n > 1$ can be represented as a product of prime factors in precisely one way, up to the order of the factors.
\end{theorem}
This theorem asserts that for any integer $n$, there exists a unique set of distinct prime numbers $p_1, p_2, ..., p_k$, along with their respective exponents $\alpha_1, \alpha_2, ..., \alpha_k$, such that
\[n=p_1^{\alpha_1} p_2^{\alpha_2} ... p_k^{\alpha_k}.\]

This form of writing numbers as a product of primes is the key to the proofs delivered in this paper.
For example, we can construct all the factors (positive divisors) of a number when given its prime factorization.
Namely, they are precisely the numbers of the form
\[d=p_1^{\beta_1} p_2^{\beta_2} ... p_k^{\beta_k},\]
where $\beta_i\le\alpha_i$ for all $1\le i \le k$.




%define arithmetic functions
%Introduce phi and sigma
%Show that they are multiplicative (Here or later?)
%motivate the problem

\section{The zeta function}

%Show the infinite sum zeta and why it's equal to the infinite product
%If extra words available, show that the are infinitely many primes.
hello \cite{basel} hello!!

%Prove the bounds for our pattern

\section{An Explanation for the Patterns}

%Explain the patterns
%Show that the lower bound is approached when large enough

%DO I WANT TO DEAL WITH PRIME DENSITY? WHAT IS A LINE?
%NEW PROBLEM: prod[96]/96^2 = prod[4]/4^2. Do such things happen infinitely often?
%Probably, but it's too hard. I try to sidestep this.
\section{Appendices}


\newpage

\printbibliography

\end{document}