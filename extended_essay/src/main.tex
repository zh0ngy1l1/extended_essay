\documentclass[a4paper, 12pt]{article}

%Paragraph jumps and indentation
\setlength{\parskip}{1.6em}
\setlength{\parindent}{0.618cm}

%Border
\usepackage[left=1in, right=1in, top=1in, bottom=1in]{geometry}

%Double spacing
\usepackage{setspace}
\doublespacing

%Packages
\usepackage{amsmath}
\usepackage[dvipsnames]{xcolor}
\usepackage{mathtools}
\usepackage{amsfonts}
\usepackage{amssymb}
\usepackage{titlesec}

%Images
\usepackage{graphicx}
\graphicspath{ {./images/} }
\usepackage{wrapfig}
\usepackage{float}

%Tables
\usepackage{multirow}
\usepackage{array}
\usepackage{tabu}
\titleformat{\section}
{\normalfont\large\bfseries}{\thesection}{1em}{}
\titleformat{\subsection}
{\normalfont\large\bfseries}{\thesubsection}{1em}{}
\usepackage{tabularx} 

%Equation numbering
\counterwithin{equation}{section}
\usepackage{hyperref}
\urlstyle{same}

%Diagrams
\usepackage{pgfplots}
\pgfplotsset{compat=newest}
\usetikzlibrary{positioning, arrows.meta}
\usepgfplotslibrary{fillbetween}

%Theorems
\usepackage{amsthm}
\newtheorem{theorem}{Theorem}
\newtheorem{lemma}{Lemma}
\newtheorem{corollary}{Corollary}
\newtheorem{question}{Question}

\theoremstyle{definition}
\newtheorem{definition}{Definition}
\newtheorem*{notation}{Notation}
\newtheorem*{problem}{Problem}

%Bibliography
\usepackage[
  backend=biber
]{biblatex}
\addbibresource{refs.bib}

%Fancy footnote
\renewcommand{\thefootnote}{\fnsymbol{footnote}}

\begin{document}

\begin{titlepage}
  \begin{center}
    Subject: Mathematics\\
    \vspace*{4cm}
    {\Large \textbf{Investigating patterns in the product of Euler's totient function $\varphi$ and the sum of divisors function $\sigma$}}\\

    \vspace{1cm}
    \textbf{Research Question:}
    What gives rise to the patterns in the sequence $a_n=n^{-2}\varphi(n)\sigma(n)$?\\

    \vfill
    \today\\
    Word count: 1729
  \end{center}
\end{titlepage}

\tableofcontents

\newpage

\section{Introduction}
Write this when most of the essay is finished
%As the British mathematician G. H. Hardy most eloquently argues in his essay \textit{A Mathematician's Apology} (1940)\cite{hardy}, pure mathematics is pursued for its aesthetic value despite having no practical applications.
%When encounting a result that seems plausible but is not formally proven, a mathematician cannot help being compelled to seek for a formal justification.
%The investigations done in the production of this paper are motivated by precisely this: It is a classic textbook example to show the bounds of $$

%Explanation why this topic is relevant and important to a student.
%Clearly stated the research question
%Methodology of the research (future tense)

\newpage
\section{Prerequisite Knowledge, Notation, and Definitions}
The concepts investigated in this paper are strictly elementary number theory, which means we're mostly working with positive integers.
Thus, for simplicity, you can assume all variables mentioned in this paper are positive integers, unless otherwise stated.

\begin{theorem}[Fundamental Theorem of Arithmetic]
  Every positive integer has an unique prime factorization.
\end{theorem}

This theorem is crucial in our ensuing discussion, because it allows us to express any number $n$ uniquely in terms of its $k$ prime divisors $p_1, p_2, ..., p_k$, along with their respective exponents $\alpha_1, \alpha_2, ..., \alpha_k$, as the product
$p_1^{\alpha_1} p_2^{\alpha_2} \dots p_k^{\alpha_k} = n$.\footnote{for $n=1$ simply set $k=0$}

\vspace{6pt} \noindent
Some more complicated sums and products are discussed in this paper, which require set-builder-like\footnote{see \url{https://en.wikipedia.org/wiki/Set-builder_notation}} notation to impose restrictions on the numbers we're summing over.
This notation will allow us to cleanly write formulas for the two functions that will be extensively discussed:

\begin{definition}[Sum of Divisors Function $\sigma$ and Euler's Totient Function $\varphi$]
  \[\sigma(n) = \sum_{d \in \mathbb{Z}^+,\ d|n} d\]
  \[\varphi(n) = \sum_{\substack{m \in \mathbb{Z}^+,\, m\leq n \\ \gcd(m,n)=1}} 1\]
\end{definition}

The comma inside the subscripts simply means `and', so verbosely we would say that
$\sigma(n)$ takes the sum of all numbers $d$ which divide $n$, and
$\varphi(n)$ counts the numbers $m\leq n$ which are coprime to $n$.

\section{The Problem of Study}
The aformentioned functions $\varphi$ and $\sigma$ are the subject of this study.
Specifically, we will solve the following problem from a classic textbook in analytic number theory\footcite[Chapter 3, Exercise 9]{apostol} and investigate it's intricacies on a deeper level.

\begin{problem}
  Show that $\displaystyle \frac{6}{\pi^2}<\frac{\varphi(n)\sigma(n)}{n^2}<1$\quad if $n\geq 2.$
\end{problem}

Luckily, this strange emergence of $\pi$ is addressed:
The textbook promises to prove, in a later chapter, that the infinite product $\prod_p (1 - p^{-2})$,
extended over all primes $p$, converges to the value $\displaystyle \frac{1}{\zeta(2)}=\frac{6}{\pi^2}$. \textit{\footnotesize Do I need to define zeta???}

In consideration of this hint, we should incorporate prime numbers to the statement.
Equally crucially, we also need to expand our understanding on these functions, as simply substituting the summations from \textbf{Definition 1} quickly becomes hopeless.
These two issues can actually be overcome simultaneously, with the following result, which characterizes $\varphi$ and $\sigma$ in terms of the prime factorization of $n$:
\begin{lemma}
  For any positive integer $n$ with prime factorization $n = \prod_{i=1}^{k} p_i^{\alpha_i}$\\
  \begin{align}
    \varphi(n) &= \prod_{i=1}^{k} p_i^{\alpha_i-1}(p_i-1)\\
    \sigma(n)  &= \prod_{i=1}^{k} \dfrac{p_i^{\alpha_i+1}-1}{p_i-1}
  \end{align}
\end{lemma}

To prove this lemma, an even more elementary result on the behaviour of $\varphi$ and $\sigma$ is required:
\begin{lemma}[Multiplicativity]
  For any numbers $m, n$ with $\gcd(m,n)=1$, we have
  \begin{align}
    \varphi(mn)=\varphi(m)\varphi(n)\\
    \sigma(mn)=\sigma(m)\sigma(n)
  \end{align}
\end{lemma}

After proving \textbf{Lemma 2} and \textbf{Lemma 1}, we proceed to solve the Problem.
Here we make an equivalent statement of the problem to facilitate the following discussion.
Define the sequence $a_n = \dfrac{\varphi(n)\sigma(n)}{n^2}$. Then the problem is equivalent to the statement $a_n \in \left(\dfrac{6}{\pi^2}, 1\right)$
A natural question that follows is whether this interval can be made narrower.
\begin{question}
  Does there exist an interval
  $\displaystyle (c^-, c^+)\subsetneq\left(\frac{6}{\pi^2},1\right)$
  such that 
  $\displaystyle c^-<a_n<c^+$
  holds for all $n\geq 2$?
\end{question}


\includegraphics[width=\textwidth]{fig1.png}
The answer turns out to be negative, which we can see with a diagram:

*Prove these statements through multiplicativity and primes.

*Whether it's before or after proving the lemma, I shall shortly hereafter motivate the product of $\varphi$ and $\sigma$: the $p_i-1$'s cancel out and we get

\[\varphi(n)*\sigma(n) = \prod_{i=1}^{k} p_i^{\alpha_i-1}(p_i^{\alpha_i+1}-1)\]
\section{The zeta function}

%Show the infinite sum zeta and why it's equal to the infinite product
%If extra words available, show that the are infinitely many primes.
hello \cite{basel} hello!!


%Prove the bounds for our pattern

\section{An Explanation for the Patterns}

%Explain the patterns
%Show that the lower bound is approached when large enough

%DO I WANT TO DEAL WITH PRIME DENSITY? WHAT IS A LINE?
%NEW PROBLEM: prod[96]/96^2 = prod[4]/4^2. Do such things happen infinitely often?
%Probably, but it's too hard. I try to sidestep this.

\section{Notes on Apostol}
Page 71 (pdf: 83): Main problem statement
Page 230 (pdf: 242): Euler product form proof.





\newpage

\printbibliography

\end{document}