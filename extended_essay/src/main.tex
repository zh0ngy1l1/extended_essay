\documentclass[a4paper, 12pt]{article}

%Paragraph jumps and indentation
\setlength{\parskip}{1.6em}
\setlength{\parindent}{1.25cm}

%Border
\usepackage[left=1in, right=1in, top=1in, bottom=1in]{geometry}

%Double spacing
\usepackage{setspace}
\doublespacing

%Packages
\usepackage{amsmath}
\usepackage[dvipsnames]{xcolor}
\usepackage{mathtools}
\usepackage{amsfonts}
\usepackage{titlesec}

%Images
\usepackage{graphicx}
\graphicspath{ {./images/} }
\usepackage{wrapfig}
\usepackage{float}

%Tables
\usepackage{multirow}
\usepackage{array}
\usepackage{tabu}
\titleformat{\section}
{\normalfont\large\bfseries}{\thesection}{1em}{}
\titleformat{\subsection}
{\normalfont\large\bfseries}{\thesubsection}{1em}{}

%Equation numbering
\counterwithin{equation}{section}
\usepackage{hyperref}
\urlstyle{same}

%Diagrams
\usepackage{pgfplots}
\pgfplotsset{compat=newest}
\usetikzlibrary{positioning, arrows.meta}
\usepgfplotslibrary{fillbetween}

\begin{document}

\begin{titlepage}
    \begin{center}
    Subject: Mathematics\\
    \vspace*{4cm}
    {\Large \textbf{Investigating patterns in the product of Euler's totient function $\varphi$ and the sum of divisors function $\sigma$}}\\

    \vspace{1cm}
    \textbf{Research Question:}
    What gives rise to the patterns and bounds in the graph of $y=\varphi(x) \cdot \sigma(x)$, and how does this relate to the Riemann zeta function and the Basel problem?\\

    \vspace{4cm}
    Word count: 237
    \vfill
    \vspace{0.1cm}
    Zhongyi Li, M26
    \end{center}
\end{titlepage}


\section*{Abstract / Preliminary Research Plan}

I will first define the functions $\varphi$ and $\sigma$, and provide motivations and justifications to why they are important and an integral part of number theory.

\noindent Secondary source for this: \textbf{Introduction to Analytic Number Theory} by \textit{Tom M. Apostol}

\begin{wrapfigure}{L}{0.5\textwidth}
    \begin{center}
        \includegraphics[width=7.9cm]{2_product.png}
        \caption{The graph I wish to investigate}
    \end{center}
\end{wrapfigure}
\noindent Next I will present the graph I plan to study, $y=\varphi(x) \cdot \sigma(x)$, and suggest computing bounds for this function.

I might hint towards patterns in the graph already at this point, but I should leave the patterns until the end, for coherence.
Instead, I introduce the upper bound of $x^2$ and lower bound of $\dfrac{x^2}{\zeta(2)}=x^2\cdot\dfrac{6}{\pi^2}$

I should go on to prove these bounds, but perhaps before that I must allude to the famous Basel Problem.


\[
\frac{1}{1^2}+\frac{1}{2^2}+\frac{1}{3^2}+\frac{1}{4^2}+... = \frac{\pi^2}{6}
\]

I will discuss the relationship of this with the Riemann Zeta function, and perhaps see the need to refer to secondary sources here.
\textbf{Modern Olympiad Number Theory} by \textit{Aditya Khurmi} is an ok choice for this, but there may exist better options.

After that, the bounds will be established and we may move on to discussing the patterns in this graph.
\begin{wrapfigure}{L}{0.7\textwidth}
    \begin{center}
        \includegraphics[width=11cm]{3_pattern.png}
        \caption{We see that the graph has denser regions when enough data is included}
    \end{center}
\end{wrapfigure}

\noindent The patterns can be clearly seen and I will be showing why they are manifest.
Maybe that requires me providing an exact definition on what a "dense sort of line thing" we see here is. That may require additional secondary sources.

I think this is enough content, and it will produce an analysis that I hope does not end up being too shallow nor too obstruse.

\end{document}