\documentclass[a4paper, 12pt]{article}

%Paragraph jumps and indentation
\setlength{\parskip}{1.6em}
\setlength{\parindent}{1.25cm}

%Border
\usepackage[left=1in, right=1in, top=1in, bottom=1in]{geometry}

%Double spacing
\usepackage{setspace}
\doublespacing

%Packages
\usepackage{amsmath}
\usepackage[dvipsnames]{xcolor}
\usepackage{mathtools}
\usepackage{amsfonts}
\usepackage{titlesec}

%Images
\usepackage{graphicx}
\graphicspath{ {./images/} }
\usepackage{wrapfig}
\usepackage{float}

%Tables
\usepackage{multirow}
\usepackage{array}
\usepackage{tabu}
\titleformat{\section}
{\normalfont\large\bfseries}{\thesection}{1em}{}
\titleformat{\subsection}
{\normalfont\large\bfseries}{\thesubsection}{1em}{}

%Equation numbering
\counterwithin{equation}{section}
\usepackage{hyperref}
\urlstyle{same}

%Diagrams
\usepackage{pgfplots}
\pgfplotsset{compat=newest}
\usetikzlibrary{positioning, arrows.meta}
\usepgfplotslibrary{fillbetween}

%Theorems
\newtheorem{theorem}{Theorem}[section]
\newtheorem{definition}{Definition}[section]
\newtheorem{lemma}{Lemma}[section]
\newtheorem{corollary}{Corollary}[section]

%Bibliography
\usepackage[
  backend=biber
]{biblatex}
\addbibresource{refs.bib}

%Headers and footers
\usepackage{fancyhdr}

\begin{document}

\begin{titlepage}
  \begin{center}
    Subject: Mathematics\\
    \vspace*{4cm}
    {\Large \textbf{Investigating patterns in the product of Euler's totient function $\varphi$ and the sum of divisors function $\sigma$}}\\

    \vspace{1cm}
    \textbf{Research Question:}
    What gives rise to the patterns in the graph of $f(n)=\sqrt{\varphi(n) \cdot \sigma(x)}$, and how does this relate to the Riemann zeta function?\\

    \vfill
    \today\\
    Word count: 1729
  \end{center}
\end{titlepage}

\pagestyle{fancy}
\fancyfoot[L]{What gives rise to the patterns in the graph of $f(n)=\sqrt{\varphi(n) \cdot \sigma(x)}$, and how does this relate to the Riemann zeta function?}
\fancyhead[L]{\thepage}
\section{Presentation to the Class}

$\varphi(n)$ is the ``number of positive integers $\le n$ and \textbf{coprime} to $n$.''
$\sigma(n)$ is the ``sum of the positive divisors of $n$."
%Examples:
\begin{center}
  \begin{minipage}{0.4\textwidth} % Reduce table width
    \centering
    \scriptsize % Shrink the font size of the table
    \begin{tabular}{||c|c@{}c|c@{}c||} 
      \hline
      $n$ & $\varphi(n)$ & numbers & $\sigma(n)$ & numbers\\ [0.5ex] 
      \hline\hline
      1 & 1 & 1 & 1 = & 1\\ 
      \hline
      2 & 1 & 1 & 3 = & 1+2\\
      \hline
      3 & 2 & 1,2 & 4 = & 1+3\\
      \hline
      4 & 2 & 1,3 & 7 = & 1+2+4\\
      \hline
      5 & 4 & 1,2,3,4 & 6 = & 1+5\\
      \hline
      6 & 2 & 1,5 & 12 = & 1+2+3+6\\
      \hline
      7 & 6 & 1,2,3,4,5,6 & 8 = & 1+7\\
      \hline
      8 & 4 & 1,3,5,7 & 15 = & 1+2+4+8\\
      \hline
      9 & 6 & 1,2,4,5,7,8 & 13 = & 1+3+9\\
      \hline
      10 & 4 & 1,3,7,9 & 18 = & 1+2+5+10\\ [1ex]
      \hline
    \end{tabular}
  \end{minipage}%
  \begin{minipage}{0.55\textwidth} % Increase image width
    \centering
    \includegraphics[width=\textwidth]{phigma_with_line.png} % Make image take full width of minipage
  \end{minipage}
\end{center}

\noindent We call functions like $\varphi$ and $\sigma$ \textit{``arithmetic functions''}, i.e. functions that express some property of the number.

\begin{lemma}[Multiplicativity of $\varphi$ and $\sigma$]
  For any two \textbf{coprime} integers $m$, $n$, we have $\varphi(mn)=\varphi(m)\varphi(n)$ and $\sigma(mn)=\sigma(m)\sigma(n)$.
\end{lemma}
%You can try proving these. Nice combinatorics exercise.

%There are nice patterns in this graph, which I will describe qualitatively.

This allows us to write closed formulas of $\varphi(n)$ and $\sigma(n)$ in terms of the prime factorization of $n$.
%This is also a nice exercise. See if you can figure the formulas out yourself.

\begin{corollary}
  $f(x)=\varphi(x)\cdot\sigma(x)$ is multiplicative.
\end{corollary}



\newpage

We wish to find some way of taking an average that sends the pairs of points to the line $y=n$, i.e. discover a function that reduces the amount of chaos and erraticity in this graph.

Initial guesses for ways to take an average: % explain what AM, GM, HM abbreviate.
\[\textbf{AM}:\ \frac{a+b}{2} \hspace{1cm} \textbf{GM}:\ \sqrt{ab} \hspace{1cm} \textbf{HM}:\ \dfrac{2}{\frac{1}{a} + \frac{1}{b}}\]

\includegraphics[width=\textwidth]{AMGMHM.png}

It's also possible to take some weighted version of these, to make the result closer to the target.
Investigating that may be a part of my Research, if the word limit allows for that.

I hope the conclusion is that the geometric mean is the best. % (whether `best' is most close to the target or easiest to compute)
Then I will proceed to graph the function $f(x)=\sqrt{\varphi(n) \cdot \sigma(n)}$ in more detail and notice some lines with points packed more densely.

\includegraphics[width=\textwidth]{GM_bounds.png}

Since it looks linear, we are motivated to graph the function $g(n) = \dfrac{\sqrt{\varphi(n)\cdot\sigma(n)}}{n}$ in more detail, to get a clearer view of what the lines are.


\includegraphics[width=\textwidth]{GM_normalized.png}


In the rest of the paper, I will compute the lower and upper bounds, show that they're asymptotically strict, %i.e. you can't have make the bound any stronger
and lastly explain why the patterns (denser lines) occur.




\newpage
\section{Abstract}
\newpage


\tableofcontents

\newpage


\section{Introduction}
Broadly speaking, number theory is the branch of mathematics concerned with the properties of integers.
In a school setting, this often includes topics such as divisibility rules, prime factorization, and, most importantly, the \textbf{Fundamental Theorem of Arithmetic}.

\begin{theorem}[Fundamental Theorem of Arithmetic]
  Every integer $n > 1$ can be represented as a product of prime factors in precisely one way, up to the order of the factors.
\end{theorem}
This theorem asserts that for any integer $n$, there exists a unique set of distinct prime numbers $p_1, p_2, ..., p_k$, along with their respective exponents $\alpha_1, \alpha_2, ..., \alpha_k$, such that
\[n=p_1^{\alpha_1} p_2^{\alpha_2} ... p_k^{\alpha_k}.\]

This form of writing numbers as a product of primes is the key to the proofs delivered in this paper.
For example, we can construct all the factors (positive divisors) of a number when given its prime factorization.
Namely, they are precisely the numbers of the form
\[d=p_1^{\beta_1} p_2^{\beta_2} ... p_k^{\beta_k},\]
where $\beta_i\le\alpha_i$ for all $1\le i \le k$.




%define arithmetic functions
%Introduce phi and sigma
%Show that they are multiplicative (Here or later?)
%motivate the problem

\section{The zeta function}

%Show the infinite sum zeta and why it's equal to the infinite product
%If extra words available, show that the are infinitely many primes.
hello \cite{basel} hello!!

%Prove the bounds for our pattern

\section{An Explanation for the Patterns}

%Explain the patterns
%Show that the lower bound is approached when large enough

%DO I WANT TO DEAL WITH PRIME DENSITY? WHAT IS A LINE?
%NEW PROBLEM: prod[96]/96^2 = prod[4]/4^2. Do such things happen infinitely often?
%Probably, but it's too hard. I try to sidestep this.
\section{Appendices}


\newpage

\printbibliography

\end{document}