\documentclass[a4paper, 12pt]{article}

%Paragraph jumps and indentation
\setlength{\parskip}{1.6em}
\setlength{\parindent}{1.25cm}

%Border
\usepackage[left=1in, right=1in, top=1in, bottom=1in]{geometry}

%Double spacing
\usepackage{setspace}
\doublespacing

%Packages
\usepackage{amsmath}
\usepackage[dvipsnames]{xcolor}
\usepackage{mathtools}
\usepackage{amsfonts}
\usepackage{titlesec}

%Images
\usepackage{graphicx}
\graphicspath{ {./images/} }
\usepackage{wrapfig}
\usepackage{float}

%Tables
\usepackage{multirow}
\usepackage{array}
\usepackage{tabu}
\titleformat{\section}
{\normalfont\large\bfseries}{\thesection}{1em}{}
\titleformat{\subsection}
{\normalfont\large\bfseries}{\thesubsection}{1em}{}

%Equation numbering
\counterwithin{equation}{section}
\usepackage{hyperref}
\urlstyle{same}

%Diagrams
\usepackage{pgfplots}
\pgfplotsset{compat=newest}
\usetikzlibrary{positioning, arrows.meta}
\usepgfplotslibrary{fillbetween}

%Theorems
\newtheorem{theorem}{Theorem}[section]
\newtheorem{definition}{Definition}[section]
\newtheorem{lemma}{Lemma}[section]
\newtheorem{corollary}{Corollary}[section]

%Bibliography
\usepackage[
  backend=biber
]{biblatex}
\addbibresource{refs.bib}

\begin{document}

\begin{titlepage}
  \begin{center}
    Subject: Mathematics\\
    \vspace*{4cm}
    {\Large \textbf{Investigating patterns in the product of Euler's totient function $\varphi$ and the sum of divisors function $\sigma$}}\\

    \vspace{1cm}
    \textbf{Research Question:}
    What gives rise to the patterns in the graph of $f(n)=\sqrt{\varphi(n) \cdot \sigma(x)}$, and how does this relate to the Riemann zeta function?\\

    \vfill
    \today\\
    Word count: 1729
  \end{center}
\end{titlepage}


\section*{Presentation to the Class}
We wish to find some way of taking an average that sends the pairs of points to the line $y=n$, i.e. discover a function that reduces the amount of chaos and erraticity in this graph.

Initial guesses for ways to take an average: % explain what AM, GM, HM abbreviate.
\[\textbf{AM}:\ \frac{a+b}{2} \hspace{1cm} \textbf{GM}:\ \sqrt{ab} \hspace{1cm} \textbf{HM}:\ \dfrac{2}{\frac{1}{a} + \frac{1}{b}}\]

\includegraphics[width=\textwidth]{AMGMHM.png}

It's also possible to take some weighted version of these, to make the result closer to the target.
Investigating that may be a part of my Research, if the word limit allows for that.

I hope the conclusion is that the geometric mean is the best. % (whether `best' is most close to the target or easiest to compute)
Then I will proceed to graph the function $f(x)=\sqrt{\varphi(n) \cdot \sigma(n)}$ in more detail and notice some lines with points packed more densely.

\includegraphics[width=\textwidth]{GM_bounds.png}

Since it looks linear, we are motivated to graph the function $g(n) = \dfrac{\sqrt{\varphi(n)\cdot\sigma(n)}}{n}$ in more detail, to get a clearer view of what the lines are.


\includegraphics[width=\textwidth]{GM_normalized.png}


In the rest of the paper, I will compute the lower and upper bounds, show that they're asymptotically strict, %i.e. you can't have make the bound any stronger
and lastly explain why the patterns (denser lines) occur.

\newpage


\tableofcontents

\newpage

\section{Introduction}
Write this when most of the essay is finished
%As the British mathematician G. H. Hardy most eloquently argues in his essay \textit{A Mathematician's Apology} (1940)\cite{hardy}, pure mathematics is pursued for its aesthetic value despite having no practical applications.
%When encounting a result that seems plausible but is not formally proven, a mathematician cannot help being compelled to seek for a formal justification.
%The investigations done in the production of this paper are motivated by precisely this: It is a classic textbook example to show the bounds of $$

%Explanation why this topic is relevant and important to a student.
%Clearly stated the research question
%Methodology of the research (future tense)

\section{Prerequisite Knowledge}
The reader should be familiar with mathematical notation and basic definitions like `coprime' and `divisor'.
Additionally, understanding of \textit{Fundamental Theorem of Arithmetic} is required:

\begin{theorem}[Fundamental Theorem of Arithmetic]
  Every positive integer has an unique prime factorization.
\end{theorem}

In other words, for any $n$, there exists a unique set of distinct prime numbers $p_1, p_2, ..., p_k$, along with their respective exponents $\alpha_1, \alpha_2, ..., \alpha_k$, such that
\[n=p_1^{\alpha_1} p_2^{\alpha_2} ... p_k^{\alpha_k}.\]
Representing a number in this form will play an exceptionally important role in this paper.

\section{An Introduction to Arithmetic Functions}

A special class of functions called \textit{arithmetic functions} play a vital role in the study of number theory, which is the study of positive integers and their properties.
The two arithmetic functions concerned in this paper are the following.

\begin{definition}[Sum of Divisors Function]
  Denoted by $\sigma(n)$, it returns the sum of all positive divisors of $n$.
\end{definition}

\begin{definition}[Euler's Totient Function]
  Denoted by $\varphi(n)$, it returns the number of positive integers $\le n$ and coprime to $n$.
\end{definition}

Mathematics relies on abstraction, so properties like these are encoded using functions.
For example, the property "n is a perfect number if it is equal to the sum of its proper divisors" can be expressed symbolically as:
\[S = \{n \in \mathbb{N} : \sigma(n) = 2n\}\]

Euler's totient function is best known as a part of \textit{Euler's Theorem}, upon which RSA encryption relies.
It turns out that both functions have neat closed forms in terms of the prime factorization of $n$.

\begin{lemma}\label{l1}
  The functions $\varphi$ and $\sigma$ have product forms
  \begin{align*}
    \varphi(n) &= \prod_{i=1}^{k} p_i^{\alpha_i-1}(p_i-1)\\
    \sigma(n)  &= \prod_{i=1}^{k} \frac{p_i^{\alpha_i+1}-1}{p_i-1}\\
    \text{given }n &= \prod_{i=1}^{k} p_i^{\alpha_i}.
  \end{align*}
\end{lemma}

*Prove these statements through multiplicativity and primes.

*It might be a potential danger that I'm not addressing my research question early enough, since the proof of Lemma~\ref{l1} will take about three pages and I plan to do it here.

*Whether it's before or after proving the lemma, I shall shortly hereafter motivate the product of $\varphi$ and $\sigma$: the $p_i-1$'s cancel out and we get

\[\varphi(n)*\sigma(n) = \prod_{i=1}^{k} p_i^{\alpha_i-1}(p_i^{\alpha_i+1}-1)\]
\section{The zeta function}

%Show the infinite sum zeta and why it's equal to the infinite product
%If extra words available, show that the are infinitely many primes.
hello \cite{basel} hello!!


%Prove the bounds for our pattern

\section{An Explanation for the Patterns}

%Explain the patterns
%Show that the lower bound is approached when large enough

%DO I WANT TO DEAL WITH PRIME DENSITY? WHAT IS A LINE?
%NEW PROBLEM: prod[96]/96^2 = prod[4]/4^2. Do such things happen infinitely often?
%Probably, but it's too hard. I try to sidestep this.

\section{Appendix} \label{appendix}




\newpage

\printbibliography

\end{document}