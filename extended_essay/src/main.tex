\documentclass[a4paper, 12pt]{article}

%Paragraph jumps and indentation
\setlength{\parskip}{1.6em}
\setlength{\parindent}{0.618cm}

%Border
\usepackage[left=1in, right=1in, top=1in, bottom=1in]{geometry}

%Double spacing
\usepackage{setspace}
\doublespacing

%Packages
\usepackage{amsmath}
\usepackage[dvipsnames]{xcolor}
\usepackage{mathtools}
\usepackage{amsfonts}
\usepackage{amssymb}
\usepackage{titlesec}
\usepackage{enumitem}
\usepackage{cancel}
\usepackage{hyperref}
\usepackage{cleveref}
\usepackage{changepage}

%Images
\usepackage{graphicx}
\graphicspath{ {./images/} }
\usepackage{wrapfig}
\usepackage{float}

%Tables
\usepackage{multirow}
\usepackage{array}
\usepackage{tabu}
\titleformat{\section}
{\normalfont\large\bfseries}{\thesection}{1em}{}
\titleformat{\subsection}
{\normalfont\large\bfseries}{\thesubsection}{1em}{}
\usepackage{tabularx}
\usepackage{ragged2e}

%Equation numbering
\counterwithin{equation}{section}
\usepackage{hyperref}
\urlstyle{same}

%Diagrams
\usepackage{pgfplots}
\pgfplotsset{compat=newest}
\usetikzlibrary{positioning, arrows.meta}
\usepgfplotslibrary{fillbetween}

%Theorems
\usepackage{amsthm}
\newtheorem{theorem}{Theorem}
\newtheorem{lemma}{Lemma}
\newtheorem{corollary}{Corollary}
\newtheorem{question}{Question}

\theoremstyle{definition}
\newtheorem{definition}{Definition}
\newtheorem{example}{Example}
\newtheorem{problem}{Problem}
\newtheorem*{terminology}{Terminology}
\newtheorem*{notation}{Notation}

%Refer to theorems
\Crefname{theorem}{Theorem}{Theorems}

\Crefname{lemma}{Lemma}{Lemmas}

\Crefname{corollary}{Corollary}{Corollaries}

\Crefname{question}{Question}{Questions}

\Crefname{definition}{Definition}{Definitions}

\Crefname{example}{Example}{Examples}

\Crefname{problem}{Problem}{Problems}

%Bibliography
\usepackage[
  backend=biber
]{biblatex}
\addbibresource{refs.bib}

%Fancy footnote
\renewcommand{\thefootnote}{\fnsymbol{footnote}}

\begin{document}

\begin{titlepage}
  \begin{center}
    Subject: Mathematics\\
    \vspace*{4cm}
    {\Large \textbf{Investigating patterns in the product of the sum of divisors function $\sigma$ and Euler's totient function $\varphi$}}\\

    \vspace{1cm}
    \textbf{Research Question:}
    What gives rise to the patterns in the graph of $f(n)=n^{-2}\sigma(n)\varphi(n)$?\\

    \vfill
    Word count: 1729
  \end{center}
\end{titlepage}

\tableofcontents

\newpage

\section{Introduction}
Write this when most of the essay is finished
%As the British mathematician G. H. Hardy most eloquently argues in his essay \textit{A Mathematician's Apology} (1940)\cite{hardy}, pure mathematics is pursued for its aesthetic value despite having no practical applications.
%When encounting a result that seems plausible but is not formally proven, a mathematician cannot help being compelled to seek for a formal justification.
%The investigations done in the production of this paper are motivated by precisely this: It is a classic textbook example to show the bounds of $$

%Explanation why this topic is relevant and important to a student.
%Clearly stated the research question
%Methodology of the research (future tense)

\begin{figure}[h]
  \centering
  \includegraphics[width=\linewidth]{2million_simple.png}
  \caption{what?}
\end{figure}

%%%%%%%%%%%%%%%%%%%
%%% NEW SECTION
%%%%%%%%%%%%%%%%%%%
\newpage
\section{Prerequisite Knowledge, Notation, and Definitions}
The inarguably most important theorem in number theory is the following:
\begin{theorem}[Fundamental Theorem of Arithmetic]
  Every positive integer has a unique prime factorization.
\end{theorem}
This theorem is crucial in our ensuing discussion, because it allows us to express any integer $n\geq 2$ uniquely in terms of its $k$ prime divisors $p_1, p_2, ..., p_k$,
along with the positive integers $\alpha_1, \alpha_2, ..., \alpha_k$ which indicate the number of times that $n$ is divisible by the corresponding prime factor.
With this, we obtain the familiar form

\begin{equation}
  n = p_1^{\alpha_1} p_2^{\alpha_2} \dots p_k^{\alpha_k} = \prod_{i=1}^{k}p_i^{\alpha_i}
\end{equation}

In the special case $n=1$, there are no prime factors, and by convention
\begin{equation}
  \textit{the empty product equals 1.}
\end{equation}

The reader should be comfortable with $\Pi$-notation for sequential products, which is similar to $\Sigma$ but for multiplication.

\subsection{Introduction to Sets}
A \textit{set} is an \textit{unordered} collection of elements.
The discussion in this paper only involves sets of numbers, such as $\mathbb{Z}$, used to denote the set of integers,
and $\mathbb{R}$, used to denote the set of real numbers.

Here \textit{unordered} means that for example, the set $A =\{1, 2\}$ is the same as
the set $B=\{2, 1\}$.
When two sets $A, B$ are the same, we write $A = B$.\\

\noindent\begin{notation}
  \begin{itemize}
    Let $A$, $B$, and $S$ be sets.
    \item The notation $x \in S$ means that $x$ is an element of the set $S$.
    \item For a finite set $S$, the symbol $|S|$ denotes the number of elements contained in $S$.
    %\item The union $A \cup B$ is the set of all elements that belong to $A$, to $B$, or to both.
    %\item The intersection $A \cap B$ is the set of all elements that belong to both $A$ and $B$.
    %\item The difference $A \setminus B$ is the set of all elements that belong to $A$ but not $B$.
  \end{itemize}
\end{notation}\vspace{0.5em}

\subsection{Set-builder Notation}
Set-builder notation is a way for mathematicians to define sets symbolically rather than verbosely.
Basic set-builder notation is used frequently throughout this paper.

\begin{notation}
  The set $S$ of elements in the set $A$ that fulfill a certain condition is defined, i.e. `built', with the following notation:
  \[S = \{a \in A\ :\ \text{conditions for }a\}\]
\end{notation}

In this case, $S$ is a subset of $A$, because every element of $S$ is an element of $A$.
This is denoted by $S\subseteq A$.
The reader should understand the following examples before proceeding:

\begin{example}
  Build the set of positive integers, denoted $\mathbb{Z}^+$, by choosing the integers which are positive:
  \begin{equation}
    \mathbb{Z}^+  = \{x \in \mathbb{Z}\ :\ x > 0 \}
  \end{equation}
\end{example}\vspace{0.5em}

\begin{example}
  Build the set of positive divisors of an integer $n\neq 0$, denoted $\mathcal{D}_n$, by choosing the positive integers that divide $n$:
  \begin{equation}
    \mathcal{D}_n = \{x \in \mathbb{Z}^+\ :\  x \mid n \}
  \end{equation}
  The notation $a\mid b$ means $a$ divides $b$.
\end{example}\vspace{0.5em}

\begin{example}\label{ex:divisor_set}
  Recall that prime numbers are positive integers greater than 1 whose only positive divisors are 1 and itself.
  Build the set of primes, denoted $\mathbb{P}$, using the set of positive divisors defined in (2.4):
  \begin{equation}\label{eq:divisor_set}
    \mathbb{P} = \left\{x \in \mathbb{Z}^+\ :\ x>1,\ \mathcal{D}_x = \{1,\ x\} \right\}
  \end{equation}
  In this case, multiple conditions on $x$ are necessary, and notationally we separate them with a comma.
\end{example}

More complicated sequential sums and products also require set-builder-like notation in the subscripts, used to impose restrictions on what to take the sum or product over.
(\ref{eq:sigma}) in \textbf{\Cref{def:sigma}} below serves as an example of this.

\subsection{Key Definitions}
The following functions, defined for all positive integers $n$, are the central focus of this study.

\begin{definition}[Sum of Divisors Function $\sigma$]\label{def:sigma}
  Define $\sigma(n)$ as the sum of all positive divisors of $n$.

  Mimicking set-builder notation, we can write
  \begin{equation}\label{eq:sigma}
    \sigma(n) = \sum_{d \in \mathcal{D}_n} d
  \end{equation}
\end{definition}

\begin{example}
  Compute $\sigma(12)$:
  Since the positive divisors of 12 are $\mathcal{D}_{12}=\{1, 2, 3, 4, 6, 12\}$,
  taking their sum yields $\sigma(12)=1+2+3+4+6+12=28$
\end{example}

\vspace{1em}\begin{definition}[Euler's Totient Function $\varphi$]\label{def:totient}
  Define $\varphi(n)$ as the number of positive integers less than or equal to $n$ and \textit{coprime} to $n$.
  Formally, define
  \begin{equation}\label{eq:totient}
    \varphi(n) = \left|\{x\in \mathbb{Z}^+\ :\ x\leq n,\ \gcd(x,n)=1\}\right|
  \end{equation}
\end{definition}
\vspace{1em}
\begin{terminology}
  We say that `$a$ is coprime to $b$', or `$a$ and $b$ are coprime', if $\gcd(a,b)=1$.
\end{terminology}\vspace{1em}

\begin{example}
  Compute $\varphi(12)$:

  \noindent The positive integers less than or equal to 12 and coprime to 12 are 1, 5, 7, 11.
  Since there are 4 such numbers, $\varphi(12)=4$.
\end{example}

%%%%%%%%%%%%%%%%%%%
%%% NEW SECTION
%%%%%%%%%%%%%%%%%%%
\newpage

\section{The Problem of Study}
This investigation was inspired by the following problem from the undergraduate textbook \textit{Introduction to Analytic Number Theory} \cite{apostol}.

\begin{problem}\label{prob:1}
  Show that $\displaystyle \frac{6}{\pi^2}<\sigma(n)\varphi(n) n^{-2}<1$ for all integers $n$ greater than 1.
\end{problem}

For simplicity, $f(n)$ will be used to denote $\sigma(n)\varphi(n) n^{-2}$ from this point onwards.

\begin{figure}[h]
  \centering
  \includegraphics[width=0.8\textwidth]{2500_bounds.png}
  \caption{Graph for \textbf{\Cref{prob:1}}}
  \label{fig:2500_bounds}
\end{figure}

\textbf{\Cref{fig:2500_bounds}} visualizes the scenario by plotting points $(n, f(n))$ (in blue) and the lower-- and upper bounds (in red) on the Cartesian plane.
Then, the task in \textbf{\Cref{prob:1}} is to show that all blue points fall in between the red lines.

The research question of this investigation, regarding `patterns' in the graph can now be put concretely.
In addition to solving \textbf{\Cref{prob:1}}, this paper will develop the necessary tools to answer the following questions.

\begin{question}\label{qn:1}
  Why are `denser lines' formed?
\end{question}

This question regards the horizontal lines that many of the points seem to fall in, at $y=1,\ y=0.75$, and $y=0.89$ for example.
This is one of the first questions arising upon seeing \textbf{\Cref{fig:2500_bounds}}, since the orderly patterns strike as unexpected in such a scattered and seemingly random distribution.
After addressing \textbf{\Cref{q:1}}, the next natural question is,

\begin{question}\label{qn:2}
  Where can `denser lines' form?
\end{question}

This is also a natural thing to ask; we see many dense lines between 0.95 and 1, and much fewer below 0.70.


%%%%%%%%%%%%%%%%%%%
%%% NEW SECTION
%%%%%%%%%%%%%%%%%%%
\section{Background for Problem 1}
To approach \textbf{\Cref{prob:1}}, we must improve our understanding on the functions $\sigma$ and $\varphi$.
The following product forms for these functions in terms of the prime factorization of $n$ shall be derived.

\vspace{1em}\begin{lemma}\label{lemma:prod_form}
  For any positive integer $n$ with prime factorization $n = \prod_{i=1}^{k} p_i^{\alpha_i}$ holds
  \begin{align}
    \sigma(n)  &= \prod_{i=1}^{k} \frac{p_i^{\alpha_i+1}-1}{p_i-1} \label{eq:prod_sigma}\\
    \varphi(n) &= \prod_{i=1}^{k} p_i^{\alpha_i-1}(p_i-1) \label{eq:prod_totient}
  \end{align}
\end{lemma}\vspace{1em}

\subsection{Deriving the Product Form of $\sigma$}
To do this, we first show that $\sigma$ is \textit{multiplicative}, which means
\begin{equation}
  \sigma(mn)=\sigma(m)\sigma(n)
\end{equation}
holds for all coprime positive integers $m, n$.
The following lemma is used to show \textit{multiplicativity}.

\vspace{1em}\begin{lemma}\label{lemma:sigmal1}
  For coprime positive integers $m, n$,
  each element $d$ in the set $\mathcal{D}_{mn}$ is a product of a
  unique pair of elements in $\mathcal{D}_m$ and $\mathcal{D}_n$.

  Conversely, each pair of elements in $\mathcal{D}_m$ and $\mathcal{D}_n$
  has a product which is a unique element in $\mathcal{D}_{mn}$.
\end{lemma}
\vspace{1em}
\begin{example}
  Let $m=9,\ n=14$.
  The divisors of 9 are $\mathcal{D}_9=\{1, 3, 9\}$ and of $14$ are
  $\mathcal{D}_{14}=\{1, 2, 7, 14\}$.
  Every divisor of $mn=126$ is a product of one divisor
  of $9$ and one divisor of $14$:
  \[
  \begin{aligned}
    1  &= 1\cdot1,  & 2  &= 1\cdot2,  & 7  &= 1\cdot7,  & 14  &= 1\cdot14,\\
    3  &= 3\cdot1,  & 6  &= 3\cdot2,  & 21 &= 3\cdot7,  & 42  &= 3\cdot14,\\
    9  &= 9\cdot1,  & 18 &= 9\cdot2,  & 63 &= 9\cdot7,  & 126 &= 9\cdot14.
  \end{aligned}
  \]
\end{example}\vspace{1em}

\subsubsection*{Proof of \Cref{lemma:sigmal1}}
Let $d$ be a positive divisor of $mn$, and write the prime factorization of $d$:
\begin{equation}
  d=\prod_{i=1}^{k}p_i^{\alpha_i}
\end{equation}
In this form, each term $p_i^{\alpha_i}$ either
\vspace{-0.5em}\begin{enumerate}
  \item divides $m$ and is coprime to $n$, or
  \item divides $n$ and is coprime to $m$,
\end{enumerate}\vspace{-0.5em}
because $d$ is a divisor of $mn$ and $m$, $n$ are coprime.
This means we can write
\begin{equation}
  d=d_m d_n
\end{equation}
where $d_m$ is the product of the terms which correspond
to case 1, and $d_n$ is the product of the terms
which correspond to case 2.
In this way, we guarantee that $d_m \mid m$ and $d_n \mid n$, or equivalently $d_m \in \mathcal{D}_m$ and $d_n \in \mathcal{D}_n$.

This shows that each element in $\mathcal{D}_{mn}$ is a product of some pair of elements $d_m \in \mathcal{D}_m$ and $d_n \in \mathcal{D}_n$.
The process of separating the terms $p_i^{\alpha_i}$ between case 1 and case 2 does not involve choices, so the corresponding pair $(d_m, d_n)$ is unique.

Conversely, every pair $d_m \in \mathcal{D}_m,\ d_n \in \mathcal{D}_n$ determines exactly one divisor of $mn$, namely their product $d_m d_n$.
\qed\vspace{1em}

\subsubsection*{Proving multiplicativity of $\sigma$}
Using \textbf{\Cref{lemma:sigmal1}}, \textit{multiplicativity} can be established using \textbf{\Cref{def:sigma}}.
\begin{align}
\sigma(mn) =&\sum_{d\in \mathcal{D}_{mn}} d\\
=&\sum_{d_m \in \mathcal{D}_m}\sum_{d_n \in \mathcal{D}_n} d_m d_n\\
=&\sum_{d_m \in \mathcal{D}_m} d_m \left(\sum_{d_n \in \mathcal{D}_n} d_n\right)\\
=&\left(\sum_{d_m \in \mathcal{D}_m} d_m\right)\left(\sum_{d_n \in \mathcal{D}_n} d_n\right)=\sigma(m)\sigma(n).
\end{align}
\vspace{-1em}\begin{enumerate}[label=(4.\arabic*), start=6]
  \item This is \textbf{\Cref{def:sigma}}.
  \item By \textbf{\Cref{lemma:sigmal1}}, divisors of $mn$ correspond uniquely to products $d_m d_n$ with $d_m\mid m$, $d_n\mid n$, so the sum may be rewritten as a double sum over such pairs.
  \item The inner and outer sums are independent, allowing the factors $d_m$ and $d_n$ to be separated.
  \item Since the inner sum depends only on $n$, it can be factored out of the outer sum.
\end{enumerate}

\subsubsection*{Finishing the product form of $\sigma$}
\textit{Multiplicativity} lets us express $\sigma(n)$ in terms of the prime factorization
$n=\prod_{i=1}^{k}p_i^{\alpha_i}$.
Since $p_1^{\alpha_1}, p_2^{\alpha_2}, \dots, p_k^{\alpha_k}$ are all coprime,
we may separate the terms $p_1^{\alpha_1}, p_2^{\alpha_2}, \dots, p_k^{\alpha_k}$
out one by one.
\begin{align}
  \sigma(n) = &\sigma(p_1^{\alpha_1}p_2^{\alpha_2} \dots p_k^{\alpha_k})\label{eq:sigma_apply_s}\\
  = &\sigma(p_1^{\alpha_1}) \cdot \sigma(p_2^{\alpha_2} \dots p_k^{\alpha_k})\\
  = &\sigma(p_1^{\alpha_1}) \cdot \sigma(p_2^{\alpha_2}) \cdot \sigma(p_3^{\alpha_3} \dots p_k^{\alpha_k})\\
  = &\dots\\
  = &\sigma(p_1^{\alpha_1}) \cdot \sigma(p_2^{\alpha_2}) \cdot \dots \cdot \sigma(p_k^{\alpha_k})\\
  = &\prod_{i=1}^{k}\sigma(p_i^{\alpha_i}) \label{eq:sigma_apply_e}
\end{align}

To obtain the desired form in (\ref{eq:prod_sigma}), we only need to compute $\sigma(p^\alpha)$ when $p$ is prime and $\alpha$ is any positive integer.
This is easy since
\begin{equation}
  \mathcal{D}_{p^\alpha}=\{1, p, p^2, \dots, p^\alpha\}
\end{equation}
which implies $\sigma(p^\alpha)$ is simply a geometric series
\begin{equation}
  \sigma(p^{\alpha})=1+p+p^2+\dots+p^\alpha = \frac{p^{\alpha+1}-1}{p-1}
\end{equation}
Finally, substitute this into (\ref{eq:sigma_apply_e}) to finish:
\[
  \sigma(n) = \prod_{i=1}^{k} \frac{p_i^{\alpha_i+1}-1}{p_i-1} \tag{\ref{eq:prod_sigma}}
\]
\qed\vspace{1em}

\begin{example}\label{ex:sigma_2025}
  Compute $\sigma(2025)$.
  \begin{align*}
    \sigma(2025) = &\sigma(3^4\cdot 5^2)\\
    = &\frac{3^5-1}{3-1} \frac{5^3-1}{5-1}\\
    = &121 \cdot 31 = 3751
  \end{align*}
\end{example}

\subsection{Deriving the Product Form of $\varphi$}
To do this, we first show that $\varphi$ is \textit{multiplicative}, which means
\begin{equation}
  \varphi(mn)=\varphi(m)\varphi(n)
\end{equation}
holds for all coprime positive integers $m, n$.
The following lemma, which is a special case of the \textbf{Chinese Remainder Theorem}, is used to show \textit{multiplicativity}.

\vspace{1em}\begin{lemma}\label{lemma:totientl1}
  For coprime positive integers $m, n$, each integer $x$ with $0\leq x < mn$
  has a unique pair of remainders when divided by $m$ and by $n$.

  Conversely, each pair of remainders is given by a unique integer within this range.
\end{lemma}
\vspace{1em}
\begin{example}\label{ex:remainder_12}
  Let $m=3,\ n=4$.
  Then $mn=12$, so we list all integers
  $x$ with $0 \leq x < 12$ together with their remainders
  when divided by 3 and by 4:
  \[
  \begin{aligned}
    0 &\rightarrow (0,0), & 3 &\rightarrow (0,3), & 6 &\rightarrow (0,2), & 9  &\rightarrow (0,1),\\
    1 &\rightarrow (1,1), & 4 &\rightarrow (1,0), & 7 &\rightarrow (1,3), & 10 &\rightarrow (1,2),\\
    2 &\rightarrow (2,2), & 5 &\rightarrow (2,1), & 8 &\rightarrow (2,0), & 11 &\rightarrow (2,3).
  \end{aligned}
  \]
\end{example}\vspace{1em}

\subsubsection*{Proof of \Cref{lemma:totientl1}}
We first show, by contradiction, that each pair of remainders when dividing by $m$ and by $n$ is produced by at most one integer $x$ with $0 \leq x < mn$.
Let the remainders of $x$ when divided by $m$ and by $n$ be $x_m$ and $x_n$, respectively.
Assume, for contradiction, that there exists another integer $x'$, with $0 \leq x' < mn$, which also gives remainders $x_m, x_n$.
Without loss of generality, assume $x' > x$.\footnote{This is allowed because the other case is symmetrical}

Since $x$ and $x'$ have the same remainders when divided by $m$, we have $m \mid x' - x$.
Similarly, $n \mid x' - x$.
Recall that $m$ and $n$ are coprime, these can be combined into $mn \mid x'-x$.

But $x'>x$ by assumption, so $x'-x$ is a positive multiple of $mn$, which implies
\begin{align}
  x'-x &\geq mn\\
  x'   &\geq mn + x \geq mn
\end{align}
contradicting with the assumption that $0 \leq x' \leq mn$.
Therefore, each pair of remainders is be given by at most one integer $x$ with $0\leq x < mn$.

It is known that there are $m$ possible remainders when dividing by $m$, and $n$ possible remainders when dividing by $n$.
Therefore, there are $mn$ total pairs of remainders.
This is equal to the number of integers $x$ that satisfy $0\leq x < mn$.

Since each such $x$ gives exactly one pair of remainders,
and each pair of remainders is given by at most one such $x$,
it is necessarily true that each pair of remainders is given by some such $x$.
\qed\vspace{1em}

But $\varphi(n)$ only concerns integers that are coprime to $n$.
The following lemma is necessary.

\vspace{1em}\begin{lemma}\label{lemma:totientl2}
  Given coprime positive integers $m, n$ and integer $x$ satisfying $0 \leq x < mn$,
  let the remainders of $x$ be $x_m$ and $x_n$ when divided by $m$ and by $n$, respectively.

  Then $x$ is coprime to $mn$ if and only if $x_m$ is coprime to $m$ and $x_n$ is coprime to $n$.
  Symbolically, the claim is
  \begin{equation}
    \gcd(x, mn) = 1 \iff \gcd(x_m, m) = \gcd(x_n, n) = 1
  \end{equation}
\end{lemma}
\vspace{1em}
\begin{example}

  Look back at \textbf{\Cref{ex:remainder_12}}:
  The integers $1, 5, 7, 11$ are coprime to 12.
  In the corresponding pairs of remainders, the left remainder is coprime to 3,
  and the right remainder is coprime to 4.

  Conversely, for the remaining integers $0, 2, 3, 4, 6, 8, 9, 10$,
  either the left remainder shares a prime factor with 3,
  or the right remainder shares a prime factor with 4.
\end{example}\vspace{1em}

\subsubsection*{Proof of \Cref{lemma:totientl2}}
Consider the integer division of $x$ by $m$ and by $n$:
\begin{align}
  x &= q_m m + x_m\label{eq:_eucm} \\
  x &= q_n n + x_n
\end{align}
Here $q_m$ and $q_n$ are the quotients of division by $m$ and by $n$, respectively.

If $\gcd(x,mn) > 1$, then there exists some prime $p$ which divides $x$ and divides either $m$ or $n$.
Without loss of generality, assume $p \mid m$.
Then rearrange (4.22):
\begin{equation}
  x - q_m m = x_m
\end{equation}
Both terms in the left hand side are divisible by $p$, so $x_m$ must be divisible by $p$ as well.
Hence $p$ is a common divisor of $x_m$ and $m$, implying that $\gcd(x_m, m) > 1$.

Conversely, again without loss of generality, assume that $\gcd(x_m, m) > 1$.
Then there exists some prime $p$ which divides both $x_m$ and $m$.
Since $mn$ is a multiple of $m$, it must be divisible by $p$.
Both terms in the right hand side of (\ref{eq:_eucm}) are divisible by $p$, so $x$ must be divisible by $p$ as well.
Hence $p$ is a common divisor of $x$ and $mn$, implying that $\gcd(x, mn)>1$.

We have shown that $\gcd(x,mn)>1$ holds exactly when at least one of $\gcd(x_m, m)$ and $\gcd(x_n, n)$ is greater than 1.
Equivalently, $\gcd(x,mn)=1$ holds if and only if $\gcd(x_m, m) = \gcd(x_n, n) = 1$.
\qed\vspace{1em}

\subsubsection*{Proving multiplicativity of $\varphi$}
Recall that \textbf{\Cref{def:totient}} defines $\varphi(n)$ as the number of positive integers $x\leq n$
which are coprime to $n$, while \textbf{\Cref{lemma:totientl1}} and \textbf{\Cref{lemma:totientl2}} are concerned
with nonnegative integers $x < n$.
To address this disparity, we simply note that
for any integer $n>1$, neither $n$ nor $0$ are coprime to $n$, so $\varphi(n)$ is
equal to the number of nonnegative $x < n$ which are coprime to $n$.

Since \textit{multiplicativity} is trivial for $m=1$ or $n=1$,
it is only necessary to consider coprime integers $m, n$ greater than 1.

Combining \textbf{\Cref{lemma:totientl1}} and \textbf{\Cref{lemma:totientl2}} shows that
the total number of pairs of remainders $(x_m, x_n)$ where $\gcd(x_m, m) = \gcd(x_n, n) = 1$
is equal to the number of integers $x$ satisfying $0 \leq x < mn$ and $\gcd(x, mn)=1$.
Note that this is precisely the amended definition of $\varphi(mn)$.

Moreover, since the possible remainders when dividing by $m$ are simply
${0, 1, \dots, m-1}$ and only those coprime to $m$ are counted, the number of possible
values for $x_m$ is $\varphi(m)$, again using the amended definition.
Similarly, the number of possible values for $x_n$ is $\varphi(n)$.

This means the number of pairs $(x_m, x_n)$ is $\varphi(m)\varphi(n)$.
Therefore, $\varphi(mn)=\varphi(m)\varphi(n)$.

\subsubsection*{Finishing the product form of $\sigma$}
Now, exactly like (\ref{eq:sigma_apply_s}) -- (\ref{eq:sigma_apply_e}) in the derivation of the product form of $\sigma$,
we deduce that for $n = \prod_{i=1}^{k}p_i^{\alpha_i}$ holds
\begin{equation}
  \varphi(n)=\prod_{i=1}^{k}\varphi(p_i^{\alpha_i})\label{eq:totient_apply_e}
\end{equation}

For any prime $p$ and any positive integer $\alpha$,
we may compute $\varphi(p^\alpha)$ by
counting the numbers from 1 to $p^\alpha$ and
removing all $p^{\alpha-1}$ multiples of $p$:
\begin{equation}
  \varphi(p^\alpha)=p^\alpha - p^{\alpha-1} = p^{\alpha-1}(p-1)
\end{equation}
Finally, substitute this into (\ref{eq:totient_apply_e}) to finish:
\[
  \varphi(n) = \prod_{i=1}^{k} p_i^{\alpha_i-1}(p_i-1) \tag{\ref{eq:prod_totient}}
\]
\qed\vspace{1em}

\begin{example}\label{ex:phi_2025}
  Compute $\varphi(2025)$.
  \begin{align*}
    \varphi(2025) = &\varphi(3^4\cdot 5^2)\\
    = &3^(4-1)(3-1) \cdot 5^(2-1)(5-1)\\
    = &54 \cdot 20 = 1080
  \end{align*}
\end{example}

\section{Solving Problem 1}
\begin{quote}
  \textit{`Why is pi here? And why is it squared?'}
  \hfill -- Grant Sanderson
  \end{quote}

The task was to prove $\frac{6}{\pi^2}<f(n)<1$ for all integers $n>1$.
Now that both parts of \textbf{Lemma 1} have been derived,
we can derive the following formula for $f(n) = \sigma(n)\varphi(n) n^{-2}$ in terms of the prime factorization of $n$.

\vspace{1em}\begin{lemma}\label{lemma:f_prod_form}
  For any positive integer $n$ with prime factorization $n=\prod_{i=1}^{k}p_i^{\alpha_i}$ holds
  \begin{equation}\label{eq:prod_f}
    f(n)=\prod_{i=1}^{k} \frac{p_i^{\alpha_i+1}-1}{p_i^{\alpha_i+1}}
  \end{equation}
\end{lemma}\vspace{1em}

\subsubsection*{Proof of \Cref{lemma:f_prod_form}}
Suppose that the integer $n>1$ has prime factorization $n=\prod_{i=1}^{k}p_i^{\alpha_i}$.
By (\ref{eq:prod_sigma}) and (\ref{eq:prod_totient}),
\begin{align}
  \sigma(n)\varphi(n) &= \left(\prod_{i=1}^{k} \frac{p_i^{\alpha_i+1}-1}{p_i-1}\right)
                         \left(\prod_{i=1}^{k} p_i^{\alpha_i-1}(p_i-1)\right)\\
    &= \prod_{i=1}^{k} \frac{p_i^{\alpha_i+1}-1}{\cancel{p_i-1}} p_i^{\alpha_i-1}(\cancel{p_i-1})\\
    &= \prod_{i=1}^{k} (p_i^{\alpha_i+1}-1)p_i^{\alpha_i-1} \label{eq:_sigmatotient}
\end{align}
Additionally,
\begin{align}
  n^{-2} &= \left(\prod_{i=1}^{k}p_i^{\alpha_i}\right)^{-2}\\
         &= \prod_{i=1}^{k}(p_i^{\alpha_i})^{-2}\\
         &= \prod_{i=1}^{k}p_i^{-2\alpha_i} \label{eq:_nminus2}
\end{align}
Combine (\ref{eq:_sigmatotient}) and (\ref{eq:_nminus2}) and simplify:
\begin{align}
  f(n)=\sigma(n)\varphi(n) n^{-2} &= \left(\prod_{i=1}^{k} (p_i^{\alpha_i+1}-1)p_i^{\alpha_i-1}\right)
                                \left(\prod_{i=1}^{k}p_i^{-2\alpha_i}\right)\\
    &=\prod_{i=1}^{k} (p_i^{\alpha_i+1}-1)p_i^{\alpha_i-1} p_i^{-2\alpha_i}\\
    &=\prod_{i=1}^{k} (p_i^{\alpha_i+1}-1)p_i^{\alpha_i-1-2\alpha_i}\\
    &=\prod_{i=1}^{k} (p_i^{\alpha_i+1}-1)p_i^{-\alpha_i-1}\\
    &=\prod_{i=1}^{k} \frac{p_i^{\alpha_i+1}-1}{p_i^{\alpha_i+1}}
\end{align}
\qed

\vspace{1em}\begin{example}
  Compute $f(2025)$.

  Since $2025 = 3^4\cdot 5^2$, we use (\ref{eq:prod_f}) in \textbf{\Cref{lemma:f_prod_form}}
  \[f(2025)=\frac{(3^{4+1}-1)}{3^{4+1}}\frac{(5^{2+1}-1)}{5^{2+1}} = \frac{242}{243}\cdot\frac{124}{125}=\frac{30008}{30375}\]
  Verifying using the answers of \textbf{\ref{ex:sigma_2025}} and \textbf{\ref{ex:phi_2025}}:
  \[f(2025) = \sigma(2025)\varphi(2025) 2025^{-2} = 3751\cdot1080 \cdot 2025^-2 = \frac{4051080}{4100625} = \frac{30008}{30375}\]

  The value lies between $\frac{\pi^2}{6}$ and $1$, as expected.
\end{example}\vspace{1em}

\subsection{Upper bound}
Actually, \textbf{\Cref{lemma:f_prod_form}} looks very promising in regards to solving \textbf{\Cref{prob:1}}.
In fact, the upper bound $\sigma(n)\varphi(n) n^{-2}<1$ is now obvious:
Every term in the product is strictly less than 1, so the whole product must be less than
1 whenever the product contains at least one term.
That is, whenever $n>1$.

\subsection{Lower bound}
The result $f(n)>\frac{6}{\pi^2}$ can be derived using the following lemmas:

\vspace{1em}\begin{lemma}\label{lemma:fl1}
  For any positive integer $n$ with prime factorization $n = \prod_{i=1}^{k} p_i^{\alpha_i}$, we have
  \begin{equation}
    f(n)=\prod_{i=1}^{k} \frac{p_i^{\alpha_i+1}-1}{p_i^{\alpha_i+1}} \geq \prod_{i=1}^{k} \frac{p_i^2-1}{p_i^2}
  \end{equation}
  with equality when $\alpha_1 = \alpha_2 = \dots = \alpha_k = 1$.
\end{lemma}
\vspace{1em}
\begin{lemma}[Basel Problem]\label{lemma:fl2}
  The infinite product over all primes
  $\displaystyle \prod_{p\ \text{prime}} \frac{p^2}{p^2-1}$
  is equal to the infinite sum
  $\displaystyle \sum_{n=1}^{\infty}\frac{1}{n^2}$
  and evaluates to $\displaystyle \frac{\pi^2}{6}$
\end{lemma}\vspace{1em}

\subsubsection*{Proof of \Cref{lemma:fl1}}
We compare the left product and the right product of (5.13) term by term.
For any $i$ holds
\begin{equation}
  \frac{p_i^{\alpha_i+1}-1}{p_i^{\alpha_i+1}} = 1 - \frac{1}{p_i^{\alpha_i+1}}
  \geq 1 - \frac{1}{p_i^{1+1}} = 1 - \frac{1}{p_i^2} = \frac{p_i^2-1}{p_i^2}
\end{equation}
because $\alpha_i$ is a positive integer.
Moreover, there is equality in (5.14) when $\alpha_i = 1$.

This means every term in the left product is greater or equal to the corresponding term in the right product,
so the whole left product must be greater than or equal to the whole right product.
Equality holds when all terms are equal, so $\alpha_i=1$ for all $i$.
\qed\vspace{1em}

\vspace{1em}\begin{example}
  Bound $f(2025)$ from below.

  The prime factorization is $2025 = 3^4 5^2$.
  Therefore, by \textbf{\Cref{lemma:fl1}},
  \[f(2025) \geq \frac{3^2-1}{3^2}\cdot \frac{5^2-1}{5^2} = \frac{8}{9}\cdot\frac{24}{25}=\frac{64}{75}\]

\end{example}\vspace{1em}

\subsubsection*{Proof of \Cref{lemma:fl2}}
The author will take for granted that the infinite sum $\sum_{n=1}^{\infty}\frac{1}{n^2}=\frac{\pi^2}{6}$,
and a proof will be omitted.
For the interested reader, there is a purely elementary geometrical proof\cite{basel}
popularised by the internet mathematician 3Blue1Brown\cite{Sanderson2018}.
A much shorter yet equally beautiful proof involving integration\cite{Apostol1983},
discovered by the author of \textit{Introduction to Analytic Number Theory}, is
also recommended.

This paper will only show that the infinite product over all primes and the infinite sum over positive integers are equal.

\begin{align}
  \prod_{p\ \text{prime}} \frac{p^2}{p^2-1} =& \prod_{p\ \text{prime}} \frac{1}{1-\frac{1}{p^2}}\\
  =& \prod_{p\ \text{prime}} \left(\sum_{i=0}^{\infty} \left(\frac{1}{p^2}\right)^i\right)\\
  =& \prod_{p\ \text{prime}} \left(1 + \frac{1}{p^2} + \frac{1}{p^4} + \dots\right)\\
  =& \left(1 + \frac{1}{2^2} + \frac{1}{2^4} + \dots\right)
  \left(1 + \frac{1}{3^2} + \frac{1}{3^4} + \dots\right)
  \dots \label{eq:f_inf_prod} \\
  =& 1 + \frac{1}{2^2} + \frac{1}{3^2} + \frac{1}{4^2} + \frac{1}{5^2}\dots
  = \sum_{n=1}^{\infty}\frac{1}{n^2} \label{eq:f_inf_sum}
\end{align}
\vspace{-1em}\begin{enumerate}[label=(5.\arabic*), start=15]
  \item Divide the numerator and denominator by $p^2$.
  \item Expand $\frac{1}{1-1/p^2}$ into a geometric series.
  \item Write without $\Sigma$-notation.
  \item Write without $\Pi$-notation.
  \item Explained below
\end{enumerate}
The step between (\ref{eq:f_inf_prod}) and (\ref{eq:f_inf_sum}) feels like a jump.
But each term in the infinite sum in (\ref{eq:f_inf_sum}) is produced exactly once
when the infinite product in (5.19) is expanded.
For example, the term $\tfrac{1}{4^2}$ arises from selecting $\tfrac{1}{2^4}$ in the factor corresponding to $p=2$ 
and $1$ in all the remaining factors. 
Similarly, the term $\tfrac{1}{6^2}$ arises from selecting $\tfrac{1}{2^2}$ in the factor for $p=2$, 
$\tfrac{1}{3^2}$ in the factor for $p=3$, and $1$ in all others. 
In general, each term $\tfrac{1}{n^2}$ is produced exactly once upon expanding the product,
and it corresponds to the unique prime factorisation of $n$, by the Fundamental Theorem of Arithmetic.

Therefore,
\begin{equation}
  \prod_{p\ \text{prime}} \frac{p^2}{p^2-1}
  = \sum_{n=1}^{\infty}\frac{1}{n^2}
  = \frac{\pi^2}{6}.
\end{equation}

\subsubsection*{Finishing the lower bound}
By \textbf{\Cref{lemma:fl1}}, for $n=\prod_{i=1}^k p_i^{\alpha_i}$ holds
\begin{equation}
  f(n)\geq \prod_{i=1}^k \frac{p_i^2-1}{p_i^2}
\end{equation}
Each factor $\dfrac{p^2-1}{p^2}$ is less than 1, so adding more such factors (i.e. extending the product to more primes) strictly decreases the value.
Hence
\begin{equation}
  \prod_{i=1}^k \frac{p_i^2-1}{p_i^2}
  > \prod_{p\ \text{prime}} \frac{p^2-1}{p^2} = \left(\prod_{p\ \text{prime}} \frac{p^2}{p^2-1}\right)^{-1}
\end{equation}
The equality is strict because the left product has finitely many terms, 
By \textbf{\Cref{lemma:fl2}} the product on the right equals $(\frac{\pi^2}{6})^{-1}=\frac{6}{\pi^2}$.
Therefore,
\begin{equation}
  f(n) > \frac{6}{\pi^2}.
\end{equation}
\qed\vspace{1em}

\section{Answering \Cref{qn:1}.}
Next we turn our attention from the bounds of $f(n)$ to the patterns formed in
its graph.

\begin{figure}[h]
  \begin{adjustwidth}{-1cm}{-1cm} % temporarily extend margins
    \centering
    \begin{minipage}{0.495\linewidth}
      \centering
      \includegraphics[width=\linewidth]{2500_clean.png}
      \caption{Clean graph of $(n, f(n))$}
      \label{fig:2500_clean}
    \end{minipage}
    \hfill
    \begin{minipage}{0.495\linewidth}
      \centering
      \includegraphics[width=\linewidth]{2500_densest.png}
      \caption{First six `dense lines'}
      \label{fig:2500_densest}
    \end{minipage}
  \end{adjustwidth}
  \end{figure}

Draw \textbf{\Cref{fig:2500_bounds}} again, but without the red lines indicating the bounds
to obtain \textbf{\Cref{fig:2500_clean}}.
Highlight six densest `lines' to obtain \textbf{\Cref{fig:2500_densest}}. 

Recall that the task was to explain the denser lines on which many of the points
seem to belong.




%GI)OEWHG JIGJP IWOUGJ )U$ J)IOGW
\newpage
\text{\huge \color{red}\textbackslash section\{ROUGH DRAFT TERRITORY\}}
\begin{figure}[h]
  \centering
  \begin{minipage}{0.49\linewidth}
    \centering
    \includegraphics[width=\linewidth]{2500_minmax.png}
    \caption{Plotting the Sequence}
    \label{fig:simple}
  \end{minipage}
  \hfill
  \begin{minipage}{0.49\linewidth}
    \centering
    \includegraphics[width=\linewidth]{2500_minmax.png}
    \caption{Highlighting Minima \& Maxima}
    \label{fig:minmax}
  \end{minipage}
\end{figure}

After discussing these regions with many points, we proceed to discuss the opposite:
Are there any regions with no points?
Formally, we may ask,
\begin{question}
  Does there exist a nonempty open interval
  $I\subset\left(\frac{6}{\pi^2},1\right)$
  such that
  $a_n \notin I$ holds for all $ n\geq 2$?
\end{question}

Notice how this formulation looks familiar?
Similarly to \textbf{Question 1}, the answer here is also \textbf{No.}
This is actually a generalization of that question:
Here we are showing that the sequence $a_n$ gets arbitrarily close to \textit{any} point in the interval $\left(\frac{6}{\pi^2},1\right)$,
while \textbf{Question 1} only required us to show that the sequence gets arbitrarily close to its boundary.

\newpage
Then show that the basel problem $\infty$ sum is the $\infty$ product: Page 230 (pdf: 242) of \cite{apostol}.
Then solve \textbf{Problem.} by writing $\dfrac{\sigma(n)\varphi(n)}{n^2}$ in terms of primes:
\[\frac{\sigma(n)\varphi(n)}{n^2} = \prod_{i=1}^{k} \dfrac{p_i^{\alpha_i+1}-1}{p_i^{\alpha_i+1}}\]
(we get this by plugging Lemma 1)

To finish the problem, see the maximum of each term in this product, and the minimum of each term.
Each term is at least $\dfrac{p_k^2-1}{p_k^2}$ and at most arbitrarily close to 1 but less than $1$.

Then the global maximum is strictly below 1 (we only consider $n\geq2$\footnote{maybe consider $n=1$ too but that's a matter of my taste \& preference})
Then the global minimum is strictly above $\prod_{p\in \mathbb{P}}\dfrac{p^2-1}{p^2} = \dfrac{6}{\pi^2}$ (Basel Problem)

So \textbf{Problem 1.} is done.

\section{Questions}
Next answer Question 1. Also rewrite it not using the interval because it's IB AA HL, not Topology 101.

The answer is: no, the bounds can't be uniformly improved.
For upper bound this is trivial (n = $2^k$ for very big $k$)

For lower bound it's what convergence means. Now that I think about it, Question 1 isn't really worth doing since it's trivial.

Question 2.
The topmost `dense line' is primes, (or prime powers? That would make the discussion more complete but also longer and not that much more impressive)
\begin{definition}[`dense line' should I come up with a better name :(]
  We define a subsequence $b_n$ of the sequence $a_n$ to be a `dense line' if it
\begin{itemize}
  \item increases
  \item approaches some value $t$
  \item $b_n$ = $a_g(n)$ and $g(n) < c x \ln x$\footnote{Here $x \ln x$ is the asymptotic growth of primes (also the asymptotic growth of prime powers, cool! This condition says `$b_n$ must be dense enough.')}
\end{itemize}
\end{definition}


Actually, we get a dense line $b_n = a_{d p_n}$ for any positive integer d, and the line is denser the smaller d is (obviously). here $p_n$ is the $n$th smallest prime.
Though the issue is that if I define `dense lines' like this, it's hard to verify every dense line is generated by this description.
In fact, I don't even think that's true... Maybe it is? But how do I define it better? This is the only formal definition I could come up with!

But that's enough knowledge about the dense lines that we can graph them! Is this a satisfying enough answer to Question 2?
\begin{figure}[h]
  \centering
  \begin{minipage}{0.49\linewidth}
    \centering
    \includegraphics[width=\linewidth]{2500_densest.png}
    \caption{First 6 `dense lines'}
    \label{fig:densest}
  \end{minipage}
  \hfill
  \begin{minipage}{0.49\linewidth}
    \centering
    \includegraphics[width=\linewidth]{2million_simple.png}
    \caption{plotting up to $n=2,000,000$}
    \label{fig:2million}
  \end{minipage}
\end{figure}

Now question 3. Look at figure 4. We see more blue `dense lines' than previously, but also white regions with very few points.
We are motivated to ask: Do dense lines cover everything eventually? Will there be some strip with no points whatsoever?
The answer is no, and we present an algorithm which shows this. It's really cool and relies on Nagura's bound:
\begin{theorem}
  For all $n\geq 25$ there is always a prime $p$ with $n < p < 1.2n$
\end{theorem}

And a bit of computer brute-force for $3 < n < 25$

and a tiny bit of brute-force by hand for $n=2, n=3$

After that though Nagura + induction can get the remaining cases.
I think this is my coolest result, and I only recently come up with it




\section{Concluding Remarks}
I don't know what to say here.
It's been fun investigating this, but seeing that the lines were just primes and constant multiples of primes is kinda anticlimactic for me; I like very hard problems and felt much more thrilled solving \textbf{Question 3.}

Also I need to write an introduction\dots

\newpage

\printbibliography

\end{document}