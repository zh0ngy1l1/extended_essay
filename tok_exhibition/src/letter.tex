\documentclass{letter}

\signature{S. Ramanujan}

\begin{document}
\begin{letter}{}
\opening{Dear Sir,}

I beg to introduce myself to you as a clerk in the Accounts Department of the Port Trust Office at Madras on a salary of only £20 per annum.
I am now about 23 years of age.
I have had no University education but I have undergone the ordinary school course.
After leaving school I have been employing the spare time at my disposal to work at Mathematics.
I have not trodden through the conventional regular course which is followed in a University course, but I am striking out a new path for myself.
I have made a special investigation of divergent series in general and the results I get are termed by the local mathematicians as 'startling'.

Just as in elementary mathematics you give a meaning to $a^n$ when $n$ is negative and fractional to conform to the law which holds when $n$ is a positive integer, similarly the whole of my investigations proceed on giving a meaning to Eulerian Second Integral for all values of $n$.
My friends who have gone through the regular course of University education tell me that $\int_{0}^{\infty}x^{n-1}e^{-x} \,dx = \Gamma (n)$ is true only when $n$ is positive.
They say that this integral relation is not true when $n$ is negative.
Supposing this is true only for positive values of $n$ and also supposing the definition $n \Gamma (n) = \Gamma (n+1)$ to be universally true, I have given meanings to these integrals and under the conditions I state the integral is true for all values of $n$ negative and fractional.
My whole investigations are based upon this and I have been developing this to a remarkable extent so much so that the local mathematicians are not able to understand me in my higher flights.

Very recently I came across a tract published by you styled Orders of Infinity in page 36 of which I find a statement.
that no definite expression has been as yet found for the number of prime numbers less than any given number.
I have found an expression which very nearly approximates to the real result, the error being negligible.
I would request you to go through the enclosed papers.
Being poor, if you are convinced that there is anything of value I would like to have my theorems published.
I have not given the actual investigations nor the expressions that I get but I have indicated the lines on which I proceed.
Being inexperienced I would very highly value any advice you give me.
Requesting to be excused for the trouble I give you.
\begin{flushright}
I remain, Dear Sir, Yours truly,\\
S. Ramanujan
\end{flushright}
\end{letter}
\end{document}