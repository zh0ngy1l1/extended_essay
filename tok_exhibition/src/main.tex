\documentclass[a4paper, 12pt]{article}

%Paragraph jumps and indentation
\setlength{\parskip}{1.5em}
\setlength{\parindent}{1.25cm}

%Border
\usepackage[left=1in, right=1in, top=1in, bottom=1in]{geometry}

%Double spacing
\usepackage{setspace}
\doublespacing

%Packages
\usepackage{amsmath}
\usepackage[dvipsnames]{xcolor}
\usepackage{mathtools}
\usepackage{amsfonts}
\usepackage{titlesec}

%Images
\usepackage{graphicx}
\graphicspath{ {./images/} }
\usepackage{wrapfig}
\usepackage{float}

%Tables
\usepackage{multirow}
\usepackage{array}
\usepackage{tabu}
\titleformat{\section}
{\normalfont\large\bfseries}{\thesection}{1em}{}
\titleformat{\subsection}
{\normalfont\large\bfseries}{\thesubsection}{1em}{}

%Equation numbering
\counterwithin{equation}{section}

%Links
\usepackage{hyperref}
\urlstyle{same}

%Diagrams
\usepackage{pgfplots}
\pgfplotsset{compat=newest}
\usetikzlibrary{positioning, arrows.meta}
\usepgfplotslibrary{fillbetween}
\usepackage{wrapfig}

\begin{document}

\begin{titlepage}
  \begin{center}
    \textbf{TOK EXHIBITION}\\
    \vspace*{3cm}
    \textbf{Prompt \#28: To what extent is objectivity possible in the production or acquisition of knowledge?}\\

    \vfill
    \today\\
    Word count: 211
  \end{center}
\end{titlepage}

\section{Lecture 'On Mathematical Maturity' by Thomas Garrity}
\begin{wrapfigure}{L}{0.5\textwidth}
  \begin{center}
    \includegraphics[width=7.9cm]{functions_thumbnail.jpg}
    \caption{Thumbnail of a recording$^1$}
  \end{center}
\end{wrapfigure}

Here is a recording of an excellent lecture regarding the concept of "mathematical maturity", at a summer event in 2017 for mostly undergraduates studying mathematics, and K-12 mathematics educators.
Garrity defines mathematical maturity as the ability to have an intuition on what to do when faced with a difficult or unfamiliar problem.
The lecture focuses on showing how this is an essential skill in pursuing mathematics and related subjects in academia, and in particular emphasizes that the production of mathematical knowledge \textit{relies} on subjective insights such as intuition and abstraction, even if the final proofs are rigorously objective.

Mathematics is often considered the most objective of disciplines, with its backbone being logically rigorous proofs derived from fundamentally irrefutable axioms. 
When a result is stated, proven, and verified, it will be set in stone as an objective fact, never to be supplanted by any future results that contradict it even slightly.
Throughout basic education and high school, students are constantly bombarded by examples of this kind of objectivity in their studies in mathematics.
After learning the long multiplication algorithm in primary school, it will never cease to give the objectively correct answer regardless of who, when, or where it is employed.
This imperturbable consistency is what makes it an objective fact, forever etched in the students mind together with countless other similar results.
Finally, all of this is reinforced by the assessment and feedback that students receive for their work.
Even the most unruly students know their grades in math are immune to favoritism, unlike in essay-based subjects, where flattery pays off.

This view on the purely objective results being everything in mathematics is what the lecture challenges.
While mathematical knowledge itself is indeed purely objective and impartial, it is highlighted that acquiring this knowledge, i.e. understanding and internalizing these concepts as a learner, relies on mathematical maturity, which is something quite subjective 

\vspace*{0.5cm}
\section{Gua Sha tool}

\begin{wrapfigure}{L}{0.4\textwidth}
  \begin{center}
    \includegraphics[width=6cm]{guasha.jpg}
    \caption{My Gua Sha Tool}
  \end{center}
\end{wrapfigure}
Gua Sha is a traditional treatment in Traditional Chinese Medicine (TCM) that remains widely practiced by its adherents.
The technique involves using a thin, rounded tool, often made of materials like buffalo horn, to scrape specific areas of the skin.
This scraping causes redness, which TCM interprets as the release of "toxins" trapped within the body.

The practice has remained largely unchanged for approximately 700 years.
This persistence can be attributed to the belief that longevity implies efficacy, an assumption that, while intuitive, does not necessarily hold under scientific scrutiny.
While this reasoning may seem unconvincing to those raised in a Western scientific framework, early exposure to such beliefs can make them deeply ingrained and resistant to skepticism later in life.

None of the purported benefits are substantiated by Western medicine, whereas the harmful effects, such as skin damage, are well-documented.
Nevertheless, TCM remains widely practiced today, as evidenced by my grandfather gifting me this Gua Sha tool---likely with the expectation that I will use it.

Epistemological Contrast – Compare the standards of objectivity in Western medicine (empirical, evidence-based, controlled trials) with those in TCM (historical continuity, anecdotal experience, cultural authority). This highlights how different knowledge systems define and approach objectivity.

Influence of Belief Systems – Discuss how early exposure to certain knowledge frameworks (e.g., growing up in a culture that values TCM) can shape one’s perception of objectivity. Does being raised in a specific tradition make one less likely to question its validity?

Limits of Objectivity – Acknowledge that even in scientific medicine, biases exist (e.g., pharmaceutical funding, research priorities). How does this complicate the idea that Western medicine is purely objective?

Personal Perspective – You mention receiving a Gua Sha tool from your grandfather. You could reflect on how your personal experience situates you between two knowledge systems—Western scientific skepticism and cultural tradition—and what that says about the difficulty of achieving absolute objectivity.

\end{document}