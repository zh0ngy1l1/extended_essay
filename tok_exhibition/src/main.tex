\documentclass[a4paper, 12pt]{article}

%Paragraph jumps and indentation
\setlength{\parskip}{1.5em}
\setlength{\parindent}{1.25cm}

%Border
\usepackage[left=1in, right=1in, top=1in, bottom=1in]{geometry}

%Double spacing
\usepackage{setspace}
\doublespacing

%Packages
\usepackage{amsmath}
\usepackage[dvipsnames]{xcolor}
\usepackage{mathtools}
\usepackage{amsfonts}
\usepackage{titlesec}

%Images
\usepackage{graphicx}
\graphicspath{ {./images/} }
\usepackage{wrapfig}
\usepackage{float}

%Tables
\usepackage{multirow}
\usepackage{array}
\usepackage{tabu}
\titleformat{\section}
{\normalfont\large\bfseries}{\thesection}{1em}{}
\titleformat{\subsection}
{\normalfont\large\bfseries}{\thesubsection}{1em}{}

%Equation numbering
\counterwithin{equation}{section}

%Links
\usepackage{hyperref}
\urlstyle{same}

%Diagrams
\usepackage{pgfplots}
\pgfplotsset{compat=newest}
\usetikzlibrary{positioning, arrows.meta}
\usepgfplotslibrary{fillbetween}
\usepackage{wrapfig}

\begin{document}

\begin{titlepage}
  \begin{center}
    \textbf{TOK EXHIBITION}\\
    \vspace*{3cm}
    \textbf{To what extent is objectivity possible in the production or acquisition of knowledge?}\\

    \vfill
    \today\\
    Word count: 211
  \end{center}
\end{titlepage}

\section{Lecture 'On Mathematical Maturity' by Thomas Garrity}

\begin{wrapfigure}{L}{0.5\textwidth}
  \begin{center}
    \includegraphics[width=7.9cm]{functions_thumbnail.jpg}
    \caption{Thumbnail of \href{https://youtu.be/zHU1xH6Ogs4}{\underline{\color{teal}a recording}}}
  \end{center}
\end{wrapfigure}
%\text{\empty}

{
  Here is a recording of an excellent lecture regarding the concept of "mathematical maturity", at a summer event in 2017 for undergraduates studying mathematics, and K-12 mathematics educators.
  I will simplify Garrity's definition of mathematical maturity as the ability to have an intuition on what to do when faced by an unfamiliar problem.
  The lecture focuses on showing how this is an essential skill in pursuing mathematics related subjects, and simultaneously convinces the audience that the action of making students mathematically mature is very much undervalued by educators.


  \noindent \textbf{\large Key Points of Connection:}\\
  Objectivity vs. Intuition in Mathematics:
  While mathematics is often considered the most objective of disciplines, Garrity's emphasis on intuition, abstraction, and problem-approach diversity challenges this perception. 
  The path to understanding mathematical concepts often involves subjective insights, even if the final proofs are rigorously objective.

  Subjective Pathways to Objective Knowledge:
  Developing mathematical maturity involves more than simply following objective rules. 
  It requires personal engagement, critical thinking, and creative problem-solving—all of which are subjective elements that shape the acquisition of knowledge.

  Role of Abstraction:
  Garrity mentions abstraction as key to mathematical understanding. 
  Abstraction itself involves a human-driven process of selecting which elements of a problem are essential, which introduces subjective judgment.

  Critical Thinking and Perspectives:
  Approaching problems from multiple perspectives, as Garrity suggests, reflects the idea that objectivity in knowledge acquisition may depend on reconciling subjective viewpoints.

  In your exhibition, you could argue that mathematical knowledge, despite its reputation for objectivity, involves subjective human processes that shape how knowledge is acquired and even what is considered worth studying. 
  Mathematical maturity embodies the interplay between these subjective processes and the pursuit of objective truths.
}

\section{Gua Sha tool}

\begin{wrapfigure}{L}{0.5\textwidth}
  \begin{center}
    \includegraphics[width=7.9cm]{functions_thumbnail.jpg}
    \caption{here is the caption.}
  \end{center}
\end{wrapfigure}
%\text{\empty}

\end{document}