\documentclass[a4paper, 12pt]{article}

%Paragraph jumps and indentation
\setlength{\parskip}{1.5em}
\setlength{\parindent}{1.25cm}

%Border
\usepackage[left=1in, right=1in, top=1in, bottom=1in]{geometry}

%Double spacing
\usepackage{setspace}
\doublespacing

%Packages
\usepackage{amsmath}
\usepackage[dvipsnames]{xcolor}
\usepackage{mathtools}
\usepackage{amsfonts}
\usepackage{titlesec}

%Images
\usepackage{graphicx}
\graphicspath{ {./images/} }
\usepackage{wrapfig}
\usepackage{float}

%Tables
\usepackage{multirow}
\usepackage{array}
\usepackage{tabu}
\titleformat{\section}
{\normalfont\large\bfseries}{\thesection}{1em}{}
\titleformat{\subsection}
{\normalfont\large\bfseries}{\thesubsection}{1em}{}

%Equation numbering
\counterwithin{equation}{section}

%Links
\usepackage{hyperref}
\urlstyle{same}

%Diagrams
\usepackage{pgfplots}
\pgfplotsset{compat=newest}
\usetikzlibrary{positioning, arrows.meta}
\usepgfplotslibrary{fillbetween}
\usepackage{wrapfig}

\begin{document}

\begin{titlepage}
  \begin{center}
    \textbf{TOK EXHIBITION}\\
    \vspace*{3cm}
    \textbf{Prompt \#28: To what extent is objectivity possible in the production or acquisition of knowledge?}\\

    \vfill
    \today\\
    Word count: 211
  \end{center}
\end{titlepage}

\section{Lecture 'On Mathematical Maturity' by Thomas Garrity}
\begin{wrapfigure}{L}{0.5\textwidth}
  \begin{center}
    \includegraphics[width=7.9cm]{functions_thumbnail.jpg}
    \caption{Thumbnail of a recording$^1$}
  \end{center}
\end{wrapfigure}

Here is a recording of an excellent lecture regarding the concept of "mathematical maturity", at a summer event in 2017 for mostly undergraduates studying mathematics, and K-12 mathematics educators.
Garrity defines mathematical maturity as the ability to have an intuition on what to do when faced with a difficult or unfamiliar problem.
The lecture focuses on showing how this is an essential skill in pursuing mathematics and related subjects in academia, and in particular emphasizes that the production of mathematical knowledge \textit{relies} on subjective insights such as intuition and abstraction, even if the final proofs are rigorously objective.

Mathematics is often considered the most objective of disciplines, with its backbone being logically rigorous proofs derived from fundamentally irrefutable axioms. 
When a result is stated, proven, and verified, it will be set in stone as an objective fact, never to be supplanted by any future results that contradict it even slightly.
Throughout basic education and high school, students are constantly bombarded by examples of this kind of objectivity in their studies in mathematics.
After learning the long multiplication algorithm in primary school, it will never cease to give the objectively correct answer regardless of who, when, or where it is employed.
This imperturbable consistency is what makes it an objective fact, forever etched in the students mind together with countless other similar results.
Finally, all of this is reinforced by the assessment and feedback that students receive for their work.
Even the most unruly students know their grades in math are immune to favoritism, unlike in essay-based subjects, where flattery pays off.

This view on the purely objective results being everything in mathematics is what the lecture challenges.
While mathematical knowledge itself is indeed purely objective and impartial, it is highlighted that acquiring this knowledge, i.e. understanding and internalizing these concepts as a learner, relies on mathematical maturity, which is something quite subjective 

\vspace*{0.5cm}
\section{Gua Sha tool}

\begin{wrapfigure}{L}{0.4\textwidth}
  \begin{center}
    \includegraphics[width=6cm]{guasha.jpg}
    \caption{My Gua Sha Tool}
  \end{center}
\end{wrapfigure}
Gua Sha is a traditional treatment in Traditional Chinese Medicine (TCM), a form of alternative medicine widely practiced in China.
The technique involves using a tool like this one \textbf{(Fig. 2)} to scrape specific areas of the skin, causing redness which TCM interprets as the release of ”toxins” trapped within the body.

The practice has remained unchanged for hundreds of years, due to the specious belief that longevity implies efficacy.
While this reasoning may seem unconvincing to those raised in a Western scientific framework, early exposure to such beliefs can make them deeply ingrained and resistant to skepticism later in life.

None of the purported benefits are substantiated by Western medicine, whereas the harmful effects, such as skin damage, are well-documented.
Nevertheless, TCM remains widely practiced today, as evidenced by my grandfather gifting me this Gua Sha tool ---likely with the expectation that I will use it.

In the West, a constant search for change and improvement is facilitated by the empirical and evidence-based approach on objectivity, directly contrasting the dogmatic belief on historical continuity and cultural authority held in TCM.
This conflict of beliefs makes it extremely difficult to spread our conception of scientific objectivity in medicine to traditional Chinese practitioners.
While TCM relies on historical continuity and anecdotal validation, scientific inquiry depends on falsifiability, controlled experimentation, and peer verification—methods designed to minimize personal and cultural biases.
Because of this, knowledge derived from statistical rigor and empirical testing is more aligned with objectivity than knowledge preserved through tradition alone.


Thus, while objectivity in the acquisition of knowledge is theoretically possible, it remains constrained by the epistemic frameworks through which people interpret reality.
For those who adhere to TCM, their belief system serves as a barrier to adopting the principles of scientific objectivity.
As long as alternative medicine is justified by subjective and culturally embedded standards, achieving true objectivity in medical knowledge will remain a challenge.

\vspace*{0.5cm}
\section{The First Letter of Ramanujan to Mr. G. H. Hardy}


\end{document}