\documentclass[a4paper, 12pt]{article}

%Paragraph jumps and indentation
\setlength{\parskip}{1.6em}
\setlength{\parindent}{1cm}

%Border
\usepackage[left=1in, right=1in, top=1in, bottom=1in]{geometry}

%Double spacing
\usepackage{setspace}
\doublespacing
%Packages
\usepackage{amsmath}
\usepackage[dvipsnames]{xcolor}
\usepackage{mathtools}
\usepackage{amsfonts}
\usepackage{titlesec}

%Images
\usepackage{graphicx}
\graphicspath{ {./images/} }
\usepackage{wrapfig}
\usepackage{float}

%Tables
\usepackage{multirow}
\usepackage{array}
\usepackage{tabu}
\titleformat{\section}
{\normalfont\large\bfseries}{\thesection}{1em}{}
\titleformat{\subsection}
{\normalfont\large\bfseries}{\thesubsection}{1em}{}

%Theorems
\usepackage{amsthm}
\newtheorem{theorem}{Theorem}
\newtheorem{lemma}{Lemma}
\newtheorem{corollary}{Corollary}
\newtheorem{question}{Question}

\theoremstyle{definition}
\newtheorem{definition}{Definition}
\newtheorem{example}{Example}
\newtheorem{problem}{Problem}
\newtheorem*{terminology}{Terminology}
\newtheorem*{notation}{Notation}

%Equation numbering
\counterwithin{equation}{section}

%links
\usepackage{hyperref}
\hypersetup{
    colorlinks=true,
    linkcolor=blue,
    filecolor=magenta,      
    urlcolor=cyan,
    pdftitle={Overleaf Example},
    pdfpagemode=FullScreen,
    }
\urlstyle{same}

%Refs
\usepackage{cleveref}
\Crefname{problem}{Problem}{Problems}

%Diagrams
\usepackage{pgfplots}
\pgfplotsset{compat=newest}
\usetikzlibrary{positioning, arrows.meta}
\usepgfplotslibrary{fillbetween}

%Bibliography
\usepackage[backend=biber]{biblatex}
\addbibresource{refs.bib}

\begin{document}

\begin{titlepage}
  \begin{center}
    Internal Assessment: Mathematics, Analysis and Approaches // \textbf{\Large IA PROPOSAL!!!}\\
    \vspace*{4cm}
    {\Large \textbf{Inversive solution to a Combinatorial Geometry problem (BW2024-15), relying on Sylvester-Gallai theorem.}}\\
    
    \vspace{1cm}
    \textbf{Research Question:}
    What is the bound for the number of \textit{baltic triangles} in a set of $N$ points on the plane?
    \vfill
    Word count: -1
  \end{center}
\end{titlepage}

\section{IA plan / proposal}
Solving the problem:
\begin{itemize}
  \item This involves introducing and deriving the Sylvester-Gallai theorem
  \item Then, introducing inversion and showing some basic properties, like bijection, and circles through the origin mapping to lines.
\end{itemize}
Maybe I could also solve \href{https://prase.cz/kalva/short/soln/sh99g2.html}{G2 from the 1999 IMO shortlist}; it also involves inversion, and is otherwise highly similar.
Then I could name the essay: \textbf{Inversion in Combinatorial Geometry}
and have the research question \textbf{How to solve combinatorial geometry problems with inversion?}
\newpage
Here are the two problems:
\begin{problem}[Baltic Way 2024]
  There is a set of $N\geq 3$ points in the plane, such that no three of them are collinear.
  Three points $A, B, C$ in the set are said to form a \textit{Baltic triangle} if no other point in the set lies on the circumcircle of triangle $ABC$.
  Assume that there exists at least one Baltic triangle.
  
  \noindent Show that there exist at least $\frac{N}{3}$ Baltic Triangles.
\end{problem}

\vspace{1em}\begin{problem}[IMO shortlist 1999] \label{prob:2}
  A circle is called a \textit{separator} for a set of five points in a plane
  if it passes through three of these points, it contains a fourth point in its
  interior, and the fifth point outside the circle.

  \noindent Prove that every set of five points such that no three are collinear and no
  four are concyclic has exactly four separators.
\end{problem}

I can solve both these problems easily, alongside showing Sylvester and inversion lemmas.
But if my task is to research, i.e. Extend \Cref{prob:2} to more points or maybe to the third or more dimensions.
This might be fun too, but I want to do as little as I can.
%\section{Conclusions}
\newpage
\printbibliography

\end{document}