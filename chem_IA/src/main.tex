\documentclass[a4paper, 12pt]{article}

%Paragraph jumps and indentation
\setlength{\parskip}{1.6em}
\setlength{\parindent}{1cm}

%Border
\usepackage[left=1in, right=1in, top=1in, bottom=1in]{geometry}

%Double spacing
\usepackage{setspace}
\doublespacing

%Packages
\usepackage{amsmath}
\usepackage[dvipsnames]{xcolor}
\usepackage{mathtools}
\usepackage{amsfonts}
\usepackage{titlesec}
\usepackage{mhchem}

%Images
\usepackage{graphicx}
\graphicspath{ {./images/} }
\usepackage{wrapfig}
\usepackage{float}

%Tables
\usepackage{multirow}
\usepackage{array}
\usepackage{tabu}
\titleformat{\section}
{\normalfont\large\bfseries}{\thesection}{1em}{}
\titleformat{\subsection}
{\normalfont\large\bfseries}{\thesubsection}{1em}{}

%Equation numbering
\counterwithin{equation}{section}

%links
\usepackage{hyperref}
\hypersetup{
    colorlinks=true,
    linkcolor=blue,
    filecolor=magenta,      
    urlcolor=cyan,
    pdftitle={Overleaf Example},
    pdfpagemode=FullScreen,
    }
\urlstyle{same}

%Diagrams
\usepackage{pgfplots}
\pgfplotsset{compat=newest}
\usetikzlibrary{positioning, arrows.meta}
\usepgfplotslibrary{fillbetween}

%Theorems
\newtheorem{theorem}{Theorem}[section]
\newtheorem{definition}{Definition}[section]
\newtheorem{lemma}{Lemma}[section]
\newtheorem{corollary}{Corollary}[section]

%Bibliography
\usepackage[backend=biber]{biblatex}
\addbibresource{refs.bib}

\newcommand{\diff}{\frac{\mathrm{d}}{\mathrm{d}t}}



\begin{document}

\begin{titlepage}
  \begin{center}
    Internal Assessment: Chemistry\\
    \vspace*{4cm}
    {\Large \textbf{Investigating the relationship between the concentration of \ce{MnO2} catalyst powder and the first order rate constant in the decomposition of \ce{H2O2} using measurements of pressure.}}\\
    
    \vspace{1cm}
    \textbf{Research Question:}
    How does changing the concentration of solid \ce{MnO2} catalyst powder suspended in aqueous \ce{1.1\% (v/v) H2O2 (aq)} affect the first order rate constant for its decomposition at $24^\circ$C?
    \vfill
    \today\\
    Word count: 123
  \end{center}
\end{titlepage}

\section{Introduction}

Hydrogen peroxide is widely used in disinfection, bleaching, and environmental remediation.
Its catalytic decomposition improves its oxidizing ability and is especially relevant in green chemistry due to the reduced waste generated.
Transition metal ions such as manganese, iron, cobalt, and lead are commonly used to catalyze this reaction.\cite{Abuissa}
Manganese, in the form of \ce{MnO2}, is the catalyst of choice in this investigation because it is available in the school laboratory, being stable, inexpensive, and non-toxic in small quantities.

This investigation aims to establish a quantitative relation between the concentration of manganese oxide powder and the rate of decomposition of \ce{H2O2}, by measuring the concentration of \ce{H2O2} indirectly through the amount of oxygen gas released.
Theory states that for low amounts of \ce{MnO2}, the relationship between its mass and the observed first order rate constant.

\section{Background Information}

Most existing studies use \ce{MnO2} catalysts immobilized on solid supports\cite{Do} or embedded in porous materials\cite{Kim}.
In contrast, this investigation uses fine \ce{MnO2} powder suspended in solution.
This setup presents practical limitations, most notably that the catalyst in its powdered form cannot be directly recovered after the reaction, necessitating an additional separation process.
Nonetheless, the powdered form offers simplicity for a school laboratory.

Most existing studies monitor the reaction utilizing methods such as titration on samples taken at specific intervals\cite{Abuissa} or colorimetric methods\cite{Do}, both of which allow direct measurement of the concentration of hydrogen peroxide in the solution.
Instead, this experiment opts to evaluate the progress of the reaction by measuring the amount of oxygen gas released by the reaction over time.
This can be calculated with information about the pressure of the sealed reaction vessel over time, assuming constant temperature.
Gas pressure will be measured using a pressure sensor, allowing for precise, continuous data collection suitable for computer processing.

\subsection{Relevant Theory}
\begin{quote}
  \hspace{-1em}\vrule width 1pt\hspace{1em}%
  \begin{minipage}[t]{0.95\linewidth}
  ``The rate limiting reaction is believed to be the initial reaction between hydrogen peroxide and the iron oxides \cite{Miller1995,ValentineWang1998,Miller1999}.
  Additionally, the results reveal that the decomposition of \ce{H2O2} follows pseudo-first order kinetics:
  \begin{equation}
    -\dfrac{\mathrm{d}[\ce{H2O2}]}{\mathrm{d}t} = k_\mathrm{app}[\ce{H2O2}]
  \end{equation}
  and thus:
  \begin{equation}
    \ln\left(\dfrac{[\ce{H2O2}]}{[\ce{H2O2}]_0}\right) = -k_\mathrm{app}t
  \end{equation}
  where $k_\mathrm{app}$ is the apparent first order rate constant,
  and $[\ce{H2O2}]$ and $[\ce{H2O2}]_0$ are the concentrations of \ce{H2O2} in the solution at any time $t$ and time zero, respectively.\cite{Huang}''
  \end{minipage}
\end{quote}

\noindent The data presented in \textbf{Section 4} agrees with theory, that the reaction is pseudo-first order.
The aim of this investigation is to measure how $k_\mathrm{app}$ changes in relation to the amount of catalyst used.

Now a general expression for pressure as a function of time  will be mathematically derived from (2.2).
\begin{center}
\begin{tabular}{|c >{\raggedright\arraybackslash}p{5cm}|c >{\raggedright\arraybackslash}p{5cm}|}
  \hline
  \multicolumn{4}{|c|}{Notation} \\
  \hline
  $n(\ce{H2O2})$&\ce{H2O2} in solution at time $t$&$P$&Pressure at time $t$\\
  $n(\ce{O2})$&\ce{O2} released up to $t$ &$P_0$&Initial pressure\\
  $n_0$&Initial \ce{H2O2} in solution&$P_\infty$&Final pressure\\
  $n_\mathrm{air}$&Initial air molecules&$V_l$&Solution volume\\
  $R$&Ideal gas constant&$V_f$&Flask volume\\
  $T$&Temperature&$V_g$&$= V_f-V_l$, Gas volume\\
  \hline
\end{tabular}
\end{center}

\noindent The catalysed decomposition reaction taking place is described by
\[\ce{2H2O2 (aq) ->[MnO2] 2H2O (l) + O2 (g)}\]
We use this to compute $n(\ce{H2O2})$ in terms of $n(\ce{O2})$:
\begin{equation}
  n(\ce{O2}) =2(n_0 - n(\ce{H2O2})) \implies n(\ce{H2O2}) =n_0 - \frac{n(\ce{O2})}{2}
\end{equation}
By the definition of concentration,
\[[\ce{H2O2}] = n(\ce{H2O2}) \cdot V_l^{-1}\]
Substituting (2.3) into the above yields
\begin{align}
  [\ce{H2O2}] &= \left(n_0 - \frac{n(\ce{O2})}{2}\right) V_l^{-1}\\
  [\ce{H2O2}]_0 &= n_0 \cdot V_l^{-1}
\end{align}
The ideal gas law states
\[PV_g=(n_\mathrm{air} + n(\ce{O2}))RT\]
Solve for $n(\ce{O2})$ in this expression:
\begin{equation}
  n(\ce{O2})=PV_g R^{-1} T^{-1} - n_\mathrm{air}
\end{equation}
Substituting this into (2.4) yields
\begin{equation}
  \begin{split}
    [\ce{H2O2}] &= \left(n_0 - \frac{n(\ce{O2})}{2}\right) V_l^{-1}\\
  &= \left(n_0 - \frac{PV_g R^{-1} T^{-1} - n_\mathrm{air}}{2}\right) V_l^{-1}\\
  &= \frac{1}{2}\left(2n_0 + n_\mathrm{air} - PV_g R^{-1} T^{-1}\right) V_l^{-1}
  \end{split}
\end{equation}
Notice that when all $\ce{H2O2}$ is decomposed, $n(\ce{H2O2})=0$ and (2.3) yields $n(\ce{O2})=2n_0$.
Hence $2n_0$ is the final amount of $\ce{O2}$ produced.
Then $2n_0 + n_\mathrm{air}$ can be interpreted as the final amount of gas present, so by the ideal gas law,
\begin{equation}
  (2n_0 + n_\mathrm{air})RT = P_\infty V_g
  \implies 2n_0 + n_\mathrm{air} = P_\infty V_g R^{-1} T^{-1}
\end{equation}
Substituting this into (2.7) gives
\begin{equation}
  \begin{split}
    [\ce{H2O2}] &= \frac{1}{2}\left(P_\infty V_g R^{-1} T^{-1} - PV_g R^{-1} T^{-1}\right) V_l^{-1}\\
    &= \frac{1}{2}\left(P_\infty - P\right)V_g R^{-1} T^{-1} V_l^{-1}
  \end{split}
\end{equation}

Finally, substitute (2.9) and (2.5) into (2.2).
\begin{align*}
  -k_\mathrm{app}t &= \ln\left(\dfrac{[\ce{H2O2}]}{[\ce{H2O2}]_0}\right)\\
  &= \ln\left(\dfrac{\frac{1}{2}\left(P_\infty - P\right)V_g R^{-1} T^{-1} V_l^{-1}}{n_0 \cdot V_l^{-1}}\right)\\
  &= \ln\left(\left(P_\infty - P\right) \cdot \dfrac{V_g}{2 n_0 RT}\right)
\end{align*}
Solve for $P$ to finish:
\begin{align*}
  e^{-k_\mathrm{app}t} &= (P_\infty - P)\\
  \frac{2 n_0 RT}{V_g} \cdot e^{-k_\mathrm{app}t} &= P_\infty - P\\
  P &= P_\infty - \frac{2 n_0 RT}{V_g} \cdot e^{-k_\mathrm{app}t}
\end{align*}

For clarity, set
\begin{equation}
  c = 2 n_0 RT V_g^{-1}
\end{equation}

Notice that $c$ is constant over time assuming that the temperature and volume do not change throughout the reaction.
Then the pressure at time $t$ always satisfies the following equation, for constants $P_\infty$, $c$, and $k_\mathrm{app}$.
\begin{equation}
  P = P_\infty - c\cdot e^{-k_\mathrm{app}t}
\end{equation}

\subsection{Existing Results}

For porous \ce{MnO2} on carbon nanotube (CNT) supports, the highest reported \textit{specific} rate constant is 
\[
k_\mathrm{CNT} = 35.48 \,\mathrm{g^{-1}\,min^{-1}},
\]
measured at $25^\circ$C in 100 mL of solution containing $1000$ppm \ce{H2O2}.\cite{Kim}
The existence of such a specific rate constant suggests that the apparent rate constant is roughly proportional with the mass of \ce{MnO2} used.

For pyrolusite (the primary manganese ore) powder, the highest reported specific rate constant is 
\[
k_\mathrm{pyro} = 0.061 \,\mathrm{min^{-1}\,mM^{-1}},
\]
measured at room temperature in 200 mL of solution containing $29.4$mM \ce{H2O2}.\cite{Do}
This suggests that the rate constant is proportional to the concentration of \ce{MnO2} in solution.

The results obtained in this investigation will be compared with these two results.
\textbf{Hypothesis.} Ideally a proportional relationship between the mass of \ce{MnO2} and $k_\mathrm{app}$ is observed in this investigation, and the result does not deviate majorly from the above values.

\section{Experimental}
\subsection{Materials \& Apparatus}
\begin{tabular}{>{\raggedright\arraybackslash}p{0.5\textwidth} >{\raggedright\arraybackslash}p{0.5\textwidth}}
  \multicolumn{2}{c}{\textbf{Materials}} \\ \hline
  $33\%$ (v/v) \ce{H2O2} (aq), diluted to $3.3\%$ & $10 \pm 0.5$mL per trial \\
  Pure distilled water & $20 \pm 0.5$mL per trial \\
  Fine \ce{MnO2} powder & $\approx 1$g total \\ \hline
  \multicolumn{2}{c}{\textbf{Equipment}} \\ \hline
  220 mL Erlenmeyer flask&Thermometer\\
  20 mL Medical syringe ($\pm 0.5$mL)&Electronic scale ($\pm 0.001$g)\\
  Vernier Pressure sensor\footnote{\url{https://www.vernier.com/product/gas-pressure-sensor/}, Accessed \today}&Rubber stopper\\
  Magnetic stir bar and stir plate&Parafilm \\\hline
\end{tabular}
\begin{wrapfigure}{l}{0.37\textwidth}
  \begin{center}
    \includegraphics[width=0.36\textwidth]{apparatus.jpg}
  \end{center}
  \caption{Apparatus}
  \label{fig:apparatus}
\end{wrapfigure}

\noindent \textbf{Pressure sensor specifications.} \\
Range: 0–140 kPa \quad Accuracy: $\pm 2$kPa \quad Sampling rate: 1 Hz \\
Data is recorded automatically using Logger Pro.

\noindent \textbf{Illustration.}
The stir bar is in the flask, which is sealed airtight with a rubber stopper wrapped in parafilm.
The tapered valve connector of the pressure sensor is wrapped in parafilm and inserted in the hole of the rubber stopper.
A minimal setup, with the aid of parafilm and pressure-rated tubing aims to minimize leakage at moderate ($\leq 140$kPa) pressure.\\

\subsection{Process}
\textbf{Before trials:} Prepare the necessary amount of \ce{H2O2} solution, diluted to $3.3\%$ (v/v).
Confirm the pressure sensor is ready for data collection and is connected to the valve connector with tubing.

\noindent \textbf{For each trial:} Add 20 mL water to the empty Erlenmeyer flask.
Measure the desired amount of \ce{MnO2} powder and add in the flask.
Confirm that the solution is at $24^\circ$C.
Add the stir bar and stir at 600 rpm.
Wrap the sides of the rubber stopper in parafilm and push it in the flask, sufficiently deeply to minimize leakage.
Draw 10 mL of $3.3\%$ \ce{H2O2} in the syringe.
Wrap the tapered area of the valve connector in parafilm.
Then, inject the contents of the syringe into the flask through the hole in the stopper, and (quickly) insert the tapered valve connector into the hole.

When pressure reaches 140kPa, carefully withdraw the valve connector from the stopper.
Confirm the temperature of the solution has remained roughly the same as before the reaction.
Then the trial is finished; reset the apparatus and repeat.

\subsection{Analytical Method}
With data on the relation between pressure and time, simply use Logger Pro's `fit curve' functionality to automatically fit a curve with the general form derived in (2.11).
\begin{figure}[h]
  \centering
  \includegraphics[width=0.9\textwidth]{Screenshot0_05.png}
  \caption{Logger Pro Auto Fit Curve for 0.05g trial}
\end{figure}

The curve will be fit on the data between 120 kPa and 140 kPa, except for the m(\ce{MnO2})=0.05g.
The beginning of data collection is avoided because it contains an unwanted artifact resulting from connecting the valve.
Using the Auto Fit functionality also automatically gives the error radius for $k_\mathrm{app}$, which is the only error among dependent variables that needs to be considered.
To establish proportionality with respect to catalyst concentration, the error radius of \ce{MnO2} mass and solution volume are also necessary.

In \textbf{Figure 2}, $A=P_\infty$, $B=-k_\mathrm{app}$, and $C=\ln(c)$, in relation to the constants in (2.11).
From this data, only the value of $k_\mathrm{app}$ is ultimately needed, but $P_\infty$ is a useful sanity check, to ensure no major systematic error is taking place.
\subsection{Considerations}

\section{Results and Discussion}
\subsection{Computing Theoretical Final Pressure}
We compute the theoretical final pressure and compare it to the experimental values given by Logger Pro, to avoid major errors.
\[
\rho(\mathrm{H_2O_2})=1.45\ \mathrm{g\;mL^{-1}},\qquad M(\mathrm{H_2O_2})=34.0\ \mathrm{g\;mol^{-1}},
\]
\begin{align*}
  &V_l = 30 \pm 1 \mathrm{ mL}\\
  \implies & V_{\mathrm{H_2O_2}}=0.33 \pm 0.011 \mathrm{ mL}\\
  \implies & m=\rho V=0.33\cdot1.45=0.4785 \pm 0.0160\mathrm{ g},
\end{align*}
\begin{equation}
  n(\mathrm{H_2O_2})=\frac{m}{M}=\frac{0.4785}{34.0}=0.0141 \pm 0.00047\mathrm{mol}
\end{equation}

This decomposes into $0.00705 \pm 0.00024$mol \ce{O2}.
Assuming constant gas volume and temperature, the ideal gas law states
\begin{equation}
  \Delta P = (\Delta n)RT V^{-1} \qquad \text{for } R \approx 8.314 L\ \mathrm{kPa}\ K^{-1}\ \mathrm{mol}^{-1}
\end{equation}

Initial conditions for the headspace gas:
\[
P_i=101\pm 1 \mathrm{kPa},\qquad
V_g=200\pm10 \mathrm{mL}=0.20\pm0.01 \mathrm{L},\qquad
T=24 \pm 1 ^\circ\mathrm{C}=297.15 \pm 1 \mathrm{K}
\]
Then numerically $\displaystyle \Delta P = \frac{0.00705 \cdot 8.314 \cdot 297.15}{0.20} \approx 88 \pm 7$kPa.
The final pressure $P_\infty$, in theory, is $P_0 + \Delta P \approx 88 + 102 = 190 \pm 8$ kPa.
This is a rough range close to which the experimental final pressure should belong.

\subsection{Data and Processing}
Below are the data and fitted curves by Logger Pro.
\begin{table}[h]
  \centering
  \begin{tabular}{|c|c|c|} 
   \hline
   m(\ce{MnO2}) & $k_\mathrm{app}$ & $P_\infty$\\ 
   \hline
   0.05g& $0.003043 \pm 0.000025 s^{-1}$& $142.35 \pm 0.18$kPa\\
   0.10g& $0.003454 \pm 0.000023 s^{-1}$& $167.30 \pm 0.24$kPa\\
   0.15g& $0.005248 \pm 0.000055 s^{-1}$& $176.15 \pm 0.48$kPa\\
   0.20g& $0.006974 \pm 0.000106 s^{-1}$& $176.66 \pm 0.70$kPa\\
   0.10g& $0.005308 \pm 0.000053 s^{-1}$& $176.98 \pm 0.46$kPa\\
   \hline
  \end{tabular}
  \caption{Trial Data From Logger Pro}
\end{table}

\newpage
\begin{figure}[h]
  \centering
  \begin{minipage}{0.49\linewidth}
    \centering
    \includegraphics[width=\linewidth]{Screenshot0_10.png}
    \caption{\small m(\ce{MnO2}) = 0.10g}
  \end{minipage}
  \hfill
  \begin{minipage}{0.49\linewidth}
    \centering
    \includegraphics[width=\linewidth]{Screenshot0_15.png}
    \caption{\small m(\ce{MnO2}) = 0.15g}
  \end{minipage}
\end{figure}
\begin{figure}[h]
  \centering
  \begin{minipage}{0.49\linewidth}
    \centering
    \includegraphics[width=\linewidth]{Screenshot0_20.png}
    \caption{\small m(\ce{MnO2}) = 0.20g}
  \end{minipage}
  \hfill
  \begin{minipage}{0.49\linewidth}
    \centering
    \includegraphics[width=\linewidth]{Screenshot0_25.png}
    \caption{\small m(\ce{MnO2}) = 0.25g}
  \end{minipage}
\end{figure}
\noindent It seems that $P_\infty$ in practice is always below the theoretical $P_\infty$ derived in \textbf{Section 4.1}.
The deviation is not major, and is likely caused by the reasons discussed in \textbf{Section 3.4}.

For each trial, we compute the concentration of \ce{MnO2}:
\[c(\ce{MnO2})=\frac{n(\ce{MnO2})}{V_l} = \frac{m(\ce{MnO2})}{M(\ce{MnO2}) V_l} = 383.4\ \mathrm{mmol}\ g^{-1}L^{-1} \cdot m(\ce{MnO2})\]
The uncertainty of this is $3.3\%$ resulting from $V_l = 30 \pm 1$mL; the uncertainty of catalyst mass is negligible.

The catalyst concentrations are shown in \textbf{Table 2}
\begin{table}[h]
  \centering
  \begin{tabular}{|c|c|c|} 
   \hline
   m(\ce{MnO2}) & c(\ce{MnO2})\\ 
   \hline
   0.05g& $19.17 \pm 0.63 \mathrm{mol} L^{-1}$\\
   0.10g& $38.34 \pm 1.27 \mathrm{mol} L^{-1}$\\
   0.15g& $57.51 \pm 1.90 \mathrm{mol} L^{-1}$\\
   0.20g& $76.68 \pm 2.53 \mathrm{mol} L^{-1}$\\
   0.10g& $95.85 \pm 3.16 \mathrm{mol} L^{-1}$\\
   \hline
  \end{tabular}
  \caption{Processed data}
\end{table}

The aim is to find a proportional relationship in between the concentrations in \textbf{Table 2} and the corresponding rate constants in \textbf{Table 1}.
Plotting these five data points of the form $(c(\ce{MnO2}),  k_\mathrm{app})$ onto the plane, \textbf{Figure 7} is obtained.
\begin{figure}[h]
  \centering
  \includegraphics[width=0.9\textwidth]{proportional.png}
  \caption{Data Points Plotted}
\end{figure}






\section{Conclusion}
\newpage

\printbibliography

\end{document}