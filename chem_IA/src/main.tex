\documentclass[a4paper, 12pt]{article}

%Paragraph jumps and indentation
\setlength{\parskip}{1.6em}
\setlength{\parindent}{1.25cm}

%Border
\usepackage[left=1in, right=1in, top=1in, bottom=1in]{geometry}

%Double spacing
\usepackage{setspace}
\doublespacing

%Packages
\usepackage{amsmath}
\usepackage[dvipsnames]{xcolor}
\usepackage{mathtools}
\usepackage{amsfonts}
\usepackage{titlesec}

%Images
\usepackage{graphicx}
\graphicspath{ {./images/} }
\usepackage{wrapfig}
\usepackage{float}

%Tables
\usepackage{multirow}
\usepackage{array}
\usepackage{tabu}
\titleformat{\section}
{\normalfont\large\bfseries}{\thesection}{1em}{}
\titleformat{\subsection}
{\normalfont\large\bfseries}{\thesubsection}{1em}{}

%Equation numbering
\counterwithin{equation}{section}

%links
\usepackage{hyperref}
\hypersetup{
    colorlinks=true,
    linkcolor=blue,
    filecolor=magenta,      
    urlcolor=cyan,
    pdftitle={Overleaf Example},
    pdfpagemode=FullScreen,
    }
\urlstyle{same}

%Diagrams
\usepackage{pgfplots}
\pgfplotsset{compat=newest}
\usetikzlibrary{positioning, arrows.meta}
\usepgfplotslibrary{fillbetween}

%Theorems
\newtheorem{theorem}{Theorem}[section]
\newtheorem{definition}{Definition}[section]
\newtheorem{lemma}{Lemma}[section]
\newtheorem{corollary}{Corollary}[section]

%Headers and footers
\usepackage{fancyhdr}



\begin{document}

\begin{titlepage}
  \begin{center}
    Internal Assessment: Chemistry\\
    \vspace*{4cm}
    {\Large \textbf{Investigating the relationship between the amount of $\mathrm{MnO_2}$ catalyst and the rate of reaction in the decomposition of $\mathrm{H_2O_2}$.}}\\

    \vfill
    \today\\
    Word count: 123
  \end{center}
\end{titlepage}

\section{Preliminary Research Plan}

\textbf{Materials:}
\begin{itemize}
  \item Water for dilution
  \item $\mathrm{H_2O_2} (l)$, will be diluted to 3\%
  \item $\mathrm{MnO_2} (s)$ powder, about 2g needed
\end{itemize}

\noindent \textbf{Apparatus:}
\begin{itemize}
  \item Addition funnel
  \item Two- or multi-neck flask
  \item Clamp and stand
  \item Pressure sensor with tubing
  \item Graduated pipette and beakers for $\mathrm{H_2O_2}$
  \item Thermometer to ensure constant initial temperature
\end{itemize}
\newpage

\begin{wrapfigure}{L}{0.5\textwidth}
  \begin{center}
    \includegraphics[width=0.5\textwidth]{Example.png}
    \caption{Example setup}
  \end{center}
\end{wrapfigure}

\noindent This image is captured from the video \textit{How to make an Oxygen Generator (MnO2/H2O2 Method)} uploaded to YouTube by NileRed, \url{https://youtu.be/eI-HMUCEJsI}.

My apparatus will be very similar to this; the only difference is that my rubber hosing will be connected to a pressure sensor.
In every trial, after checking that the initial temperature and pressure are constant, I precisely measure the amount of $\mathrm{MnO_2}$ to add to the flask.
Then I add some constant volume of $\mathrm{H_2O_2}$ through the addition funnel into the flask and close the funnel so air doesn't escape.
I will record the change in pressure over time using Logger Pro.

\includegraphics[width=\textwidth]{trial02.png}
\includegraphics[width=\textwidth]{trial005.png}
I see that the graph is approximately an inverse exponent function.
I shall use the coefficient C as the "rate of reaction" variable.



\end{document}